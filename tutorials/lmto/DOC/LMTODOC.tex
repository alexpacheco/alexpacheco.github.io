\documentclass[aps,twocolumn,a4]{revtex4}
%%%%%%%%%%%%%%%%%%%%%%%%%%%%%%%%%%%%%%%%%%%%%%%%%%%%%%%%%%%%
%                                                          %
% Compile with LATEX2e and the Revtex macros version 3.0   %
%                                                          %
% If you cannot print this in a reasonable way send me an  %
% e-mail: jepsen@and.mpi-stuttgart.mpg.de                  %
% and I will mail it to you.                               %
%                                                          %
%         Ove Jepsen MPI-FKF              July 1st. 1998   %
%                                                          %
%%%%%%%%%%%%%%%%%%%%%%%%%%%%%%%%%%%%%%%%%%%%%%%%%%%%%%%%%%%%
\begin{document}
\renewcommand{\thepage}{\roman{page}}
\title{The STUTTGART TB-LMTO-ASA program.}
\author{ R. W. Tank, O. Jepsen, and O. K. Andersen.}
\address{Max-Planck-Institut f\"ur
Festk\"orperforschung, Heisenbergstr. 1,\\D-70569 Stuttgart,
Federal Republic of Germany .}

\begin{abstract}
This document describes how to install and run the
TB-LMTO program version 47 under UNIX.

The development of this program has been managed by Ole Krogh Andersen;
the first version was constructed mostly by
Mark van Schilfgaarde during 1987-88.
Over the years the following people (in alphabetic order) have written,
corrected or made additions to the
program: Vladimir Antonov, Peter Braun,
Armin Burkhardt, Volker Eyert, Georges Krier,
Toni A. Paxton, Andrei Postnikov, Andreas Savin.
The latest major
revisions and additions are due to the people below the title.

This document was written by Ove Jepsen. Questions or comments
should be directed to him (jepsen@and.mpi-stuttgart.mpg.de).

For use of this program see the License
agreement in Section \ref{license}.

Version 47 was last updated September 1998.

This documentation was written in \LaTeX 2e and can be printed with
version 3.0 of the REVTEX macros.

The linear muffin-tin orbital (LMTO) method has been described in
numerous publications. For a description of the LMTO
method in general see Ref. 1, the tight-binding (TB)
version see Ref. 2, the down-folding technique Ref. 3,
the tetrahedron method Ref. 4, accuracy using overlapping spheres
Ref. 5, and a detailed example Ref. 6.
Some useful books are also worth mentioning:
{\it The Mathematical Theory of Symmetry in Solids}, Ref. 7,
contains among other things all the Brillouin zones, {\it
International Tables for Crystallography}, Ref. 8, contains all
crystallographic space groups and {\it Pearson's Handbook of
Crystallographic Data for Intermetallic phases}, Ref. 9, contains
structural
data for many compounds and elements.
\end{abstract}
\maketitle
%\twocolumn
\nopagebreak
\tableofcontents
\renewcommand{\thepage}{\arabic{page}}
\setcounter{page}{1}
%\input{install}
\section{Installation of the TB-LMTO programs}
\label{install}
The source code is compressed on
the distribution diskette from where this documentation was down-loaded.

The uncompression, creation of directories, and copying to directories
is done automatically as described in the READ.ME file
on the distribution disk. This can be done under MS-DOS or UNIX.

The GNUPLOT graphical package on the distribution diskette can be
used for all plots, see Section \ref{plot}.

Options are provided for creation of plot data files appropriate for
IBM Visualization Data Explorer version 2, which is available for
IBM, Sun, HP, and SGI machines.

The path name in the {\em makefile} and in all files with the extension
{\em .exec} in the lmto47 directory
(hereafter called the main directory, and directories one level
below this will be called subdirectories) has to be
changed to the appropriate name. Furthermore, the path name in
{\em getmain} and in {\em getmakefile} in the main
directory should also be changed.
Typing `make all' will then compile
and link all programs according to the rules in the {\em makefile}.
%\input{execute}
\section{Execution of the TB-LMTO programs}
\label{execute}

\begin{enumerate}
\item The basic input data is inserted in the CTRL file.
The CTRL file shown below contains the minimum amount of
information to be typed, in order to run the TB-LMTO programs. The rest
may be obtained from the default values in the program, which are
written into the CTRL file by the program.
\begin{verbatim}
HEADER  NiSe2 Pyrit (Acta Chem. Scand.
                     v. 23 p. 2325 1969)
VERS    LMASA-47
STRUC   ALAT=11.2683
        PLAT=1 0 0
             0 1 0
             0 0 1
CLASS   ATOM=NI Z=28
        ATOM=Se Z=34
SITE    ATOM=NI POS= .0  .0  .0
        ATOM=NI POS= .5  .0  .5
        ATOM=NI POS= .0  .5  .5
        ATOM=NI POS= .5  .5  .0
        ATOM=Se POS= .383  .383  .383
        ATOM=Se POS= .617  .617  .617
        ATOM=Se POS= .117  .617  .883
        ATOM=Se POS= .883  .383  .117
        ATOM=Se POS= .617  .883  .117
        ATOM=Se POS= .383  .117  .883
        ATOM=Se POS= .883  .117  .617
        ATOM=Se POS= .117  .883  .383
\end{verbatim}

The input consists of tokens, like ATOM= or Z=. These are grouped
together in categories, like CLASS. The name of a category always
starts in the first column of a
line. This will be explained in detail in Section \ref{categ}.
ALAT= is a lattice constant in a.u. (i.e. Bohr radii),
PLAT= are the lattice
translation vectors and POS= are the atomic positions. The latter two
are in Cartesian coordinates and in units of ALAT.

All Ni atoms and all Se atoms are equivalent i.e. there are two
classes. From the information in the CTRL file, the program will find
the symmetry i.e. all space group operations and the generators of the
space group.  If the symmetry of the bravais lattice is incompatible
with the symmetry of the basis, then the symmetry of the bravais
lattice is used.
Alternatively to the CTRL file above,
the symmetry generators and only one of each of
the inequivalent atoms could have been given. The program will complete
the basis:
\begin{verbatim}
HEADER  NiSe2 Pyrit (Acta Chem. Scand.
                     v. 23 p. 2325 1969)
VERS    LMASA-47
STRUC   ALAT=11.2683
        PLAT=1 0 0
             0 1 0
             0 0 1
CLASS   ATOM=NI Z=28
        ATOM=Se Z=34
SITE    ATOM=NI POS= .0  .0  .0
        ATOM=Se POS= .383  .383  .383
SYMGRP  NGEN=3 GENGRP=I R3D MX:(.5,.5,0)
\end{verbatim}

The category SYMGRP contains three tokens: NGEN= the number of
generators supplied,\\
GENGRP= the names of the generators as defined in
the Subsection `Category SYMGRP'.

A third possibility is to use the interactive program
{\em lminit.run} in the main directory. An example
is shown below, where the user's answer
to the query is indicated by the string after `--$>$' and followed
by `(Answer)'.
\begin{verbatim}

| QUERY : SPCGRP:
|         SYMBOL OR NUMBER OF SPACE GROUP
|
| Enter: 'nnn', nnn is the value for SPCGRP,
|         'a' to abort program execution.
|        ->205 (Answer)

  Space group: Pa-3       No.:205
  Crystal system: cubic
  Generators: I R3D MX::(1/2,1/2,0)

| QUERY :  atomic-units ? (else in Angstrom)
|
| Enter: 'nnn', nnn is the value for ATUNITS,
|               default value=T
|         'a' to abort program execution.
|         '/' to accept default value
|        ->/

   Now enter lattice parameters

| QUERY : A:lattice constant (atomic units)
|
| Enter: 'nnn', nnn is the value for A,
|         'a' to abort program execution.
|        ->11.2683 (Answer)

 CTRLUC:             PLAT   ALAT=11.2683
        1.00000    .00000    .00000
         .00000   1.00000    .00000
         .00000    .00000   1.00000

| QUERY: label or nuclear charge of class 1
|
| Enter: 'nnn', nnn is the value for label,
|         'a' to abort program execution.
|         'q' to quit
|        ->28 (Answer)

 Class: Ni  ; nuclear charge 28 (atom Ni)

| QUERY: X = position of Ni (in units of the
|         translation vectors a, b, c)
|
| Enter: 'nnn', nnn is the value for X,
|               default value=0. 0. 0.
|         'a' to abort program execution.
|         '/' to accept default value
|        ->/ (Answer)

| QUERY: label or nuclear charge of class 2
|
| Enter: 'nnn', nnn is the value for label,
|         'a' to abort program execution.
|         'q' to quit
|        ->34  (Answer)

 Class: Se  ; nuclear charge 34 (atom Se)

| QUERY : X = position of Se
|
| Enter: 'nnn', nnn is the value for X,
|               default value=0. 0. 0.
|         'a' to abort program execution.
|         '/' to accept default value
|        ->.383 .383 .383

| QUERY: label or nuclear charge of class 3
|
| Enter: 'nnn', nnn is the value for label,
|         'a' to abort program execution.
|         'q' to quit
|        ->q  (Answer)

 CHKSYM: Check the symmetry of the crystal:

 GENGRP: A subgroup of the space group will
	 be created from 3 generators
         The generators created 24 sym. ops.

 ADDBAS: The basis has been extended from
	 2 to  12 atoms.
         The new positions are:

      ATOM=Ni   POS=    .500   .500   .000
      ATOM=Ni   POS=    .000   .500   .500
      ATOM=Ni   POS=    .500   .000   .500
      ATOM=Se   POS=   -.383  -.383  -.383
      ATOM=Se   POS=    .117   .883   .383
      ATOM=Se   POS=    .883   .117  -.383
      ATOM=Se   POS=    .383   .117   .883
      ATOM=Se   POS=    .883   .383   .117
      ATOM=Se   POS=   -.383   .883   .117
      ATOM=Se   POS=    .117  -.383   .883

 SYMLAT: lattice invariant under 48 sym. ops.

 SYMCRY: crystal invariant under 24 sym. ops.

 SHOSYM: Bravais system : cubic
         Bravais lattice: cubic primitive
         Centring type  : P
         Crystal family : cubic
         Crystal system : cubic
         Point group    : m-3

\end{verbatim}

The resulting CTRL file looks like this:
\begin{verbatim}
HEADER    Ni4Se8, cubic primitive
SYMGRP    NGEN=3 GENGRP=I R3D MX:(.5,.5,0)
          SPCGRP=Pa-3
STRUC     ALAT=11.2683
          PLAT=1 0 0
               0 1 0
               0 0 1
CLASS     ATOM=Ni Z=28
          ATOM=Se Z=34
SITE      ATOM=Ni POS=0.000 0.000 0.000
          ATOM=Ni POS=0.500 0.500 0.000
          ATOM=Ni POS=0.500 0.000 0.500
          ATOM=Ni POS=0.000 0.500 0.500
          ATOM=Se POS=0.383 0.117 -.117
          ATOM=Se POS=-.383 -.117 0.117
          ATOM=Se POS=-.117 0.383 0.117
          ATOM=Se POS=0.117 -.383 -.117
          ATOM=Se POS=0.117 -.117 0.383
          ATOM=Se POS=-.117 0.117 -.383
          ATOM=Se POS=0.383 0.383 0.383
          ATOM=Se POS=-.383 -.383 -.383
\end{verbatim}

\item The next step is to find the size of the atomic spheres.
First muffin-tin (MT) radii are found by {\em lmhart.run} and inserted
into the CTRL file. The result of this run is shown below.
\begin{verbatim}
HEADER    Ni4Se8, cubic primitive
VERS      LMASA-47
IO        VERBOS=30 OUTPUT=LM ERR=ERR
SYMGRP    NGEN=3 GENGRP=I R3D MX:(.5,.5,0)
          SPCGRP=Pa-3
STRUC     ALAT=11.2683
          PLAT=1 0 0
               0 1 0
               0 0 1
DIM       NBAS=12 NCLASS=2 NL=3 LDIM=68
	  IDIM=40 NSYMOP=24
CLASS     ATOM=Ni Z=28 R=2.41776510
          ATOM=Se Z=34 R=2.28352037
SITE      ATOM=Ni POS=0.000 0.000 0.000
          ATOM=Ni POS=0.500 0.500 0.000
          ATOM=Ni POS=0.500 0.000 0.500
          ATOM=Ni POS=0.000 0.500 0.500
          ATOM=Se POS=0.383 0.117 -.117
          ATOM=Se POS=-.383 -.117 0.117
          ATOM=Se POS=-.117 0.383 0.117
          ATOM=Se POS=0.117 -.383 -.117
          ATOM=Se POS=0.117 -.117 0.383
          ATOM=Se POS=-.117 0.117 -.383
          ATOM=Se POS=0.383 0.383 0.383
          ATOM=Se POS=-.383 -.383 -.383
\end{verbatim}

The two tokens (R=) with the MT-radii in a.u. have been
inserted in category CLASS. Furthermore, a new class, IO, has been
inserted with the tokens: VERBOS= which specifies the amount of
output the programs print and the number of default tokens which will
be inserted in the CTRL file by the programs which updates the CTRL
file, OUTPUT= which gives a file name to which standard out will be
directed, ERR= which gives a file name to which error messages will
be directed.

\item The overlap between the WS-spheres is checked by the
{\em lmovl.run} program. 
The result written to standard out
looks like (the output has been edited to fit into this
documentation):
\begin{verbatim}
Info: impossible to reach VOL, increase OMMAX.

CELL VOLUME= 1430.78771,
SUM OF SPHERE VOLUMES=   992.45473

IB JB  CL1  CL2  DIST SUMRS  1/d 1/s1 1/s2
==========================================
 1  5  Ni   Se    .42 .48    16  27   28
------------------------------------------
 5  1  Se   Ni    .42 .48    16  28   27
 5  6  Se   Se    .41 .47    16  28   28
------------------------------------------

\end{verbatim}

The first line tells that space filling could not be reached with the
default maximum overlaps (two atomic spheres are allowed to overlap
no more than 16\% in the sence of 1/d below, max. overlap between
an atomic and an
interstitial sphere is 18\%, and max. overlap between two
interstitial spheres is 20\%).

The next two lines show the cell volume and the volume inside the
expanded spheres.

In the table,
each line displays two neighboring atoms with WS-radii s1 and s2, and
the distance between them d. Under `1/d', `1/s1', and `1/s2' are the
results of $s1+s2-d \over d$, $s1+s2-d \over s1$, and
 $s1+s2-d \over s2$, respectively, in percent.
If the two sphere radii are very much different, one should check
`1/s1', where s1 is the smallest radius. (A big sphere may swallow a
small sphere.)

Since in the present case space filling could not be reached,
interstitial spheres have to be inserted.

\item Interstitial spheres are found by the {\em lmes.run} program.
The resulting CTRL file is shown below. 
\begin{verbatim}
HEADER Ni4Se8, cubic primitive
VERS   LMASA-47
IO     VERBOS=30 OUTPUT=OVL ERR=ERR
SYMGRP NGEN=3 GENGRP=I R3D MX:(.5,.5,0)
       SPCGRP=Pa-3
STRUC  ALAT=11.2683
       PLAT=1 0 0
            0 1 0
            0 0 1
DIM    NBAS=36 NCLASS=3 NL=3 LDIM=68
       IDIM=40 NSYMOP=24
CLASS  ATOM=Ni Z=28 R=2.41776510
       ATOM=Se Z=34 R=2.28352037
       ATOM=E  Z= 0 R=1.53150056
SITE   ATOM=Ni POS=0.00000 0.00000 0.00000
       ATOM=Ni POS=0.50000 0.50000 0.00000
       ATOM=Ni POS=0.50000 0.00000 0.50000
       ATOM=Ni POS=0.00000 0.50000 0.50000
       ATOM=Se POS=0.38300 0.11700 -.11700
       ATOM=Se POS=-.38300 -.11700 0.11700
       ATOM=Se POS=-.11700 0.38300 0.11700
       ATOM=Se POS=0.11700 -.38300 -.11700
       ATOM=Se POS=0.11700 -.11700 0.38300
       ATOM=Se POS=-.11700 0.11700 -.38300
       ATOM=Se POS=0.38300 0.38300 0.38300
       ATOM=Se POS=-.38300 -.38300 -.38300
       ATOM=E  POS=0.28510 -.18060 0.09608
       ATOM=E  POS=-.28510 0.18060 -.09608
       ATOM=E  POS=0.09608 0.28510 -.18060
       ATOM=E  POS=-.09608 -.28510 0.18060
       ATOM=E  POS=0.18060 -.09608 -.28510
       ATOM=E  POS=-.18060 0.09608 0.28510
       ATOM=E  POS=0.31939 0.09608 0.21489
       ATOM=E  POS=0.21489 0.31939 0.09608
       ATOM=E  POS=-.31939 -.09608 -.21489
       ATOM=E  POS=-.21489 -.31939 -.09608
       ATOM=E  POS=0.09608 0.21489 0.31939
       ATOM=E  POS=-.09608 -.21489 -.31939
       ATOM=E  POS=0.40391 -.21489 -.18060
       ATOM=E  POS=-.40391 0.21489 0.18060
       ATOM=E  POS=-.18060 0.40391 -.21489
       ATOM=E  POS=0.18060 -.40391 0.21489
       ATOM=E  POS=0.21489 0.18060 -.40391
       ATOM=E  POS=-.21489 -.18060 0.40391
       ATOM=E  POS=0.40391 -.28510 0.31939
       ATOM=E  POS=0.31939 0.40391 -.28510
       ATOM=E  POS=-.40391 0.28510 -.31939
       ATOM=E  POS=-.31939 -.40391 0.28510
       ATOM=E  POS=-.28510 0.31939 0.40391
       ATOM=E  POS=0.28510 -.31939 -.40391
\end{verbatim}

The program found 24 interstitial spheres, all equivalent, with a sphere
radius between 0.9 and 4.0 a.u. These limits are default and can be
changed. Notice that a new category DIM has been inserted, which
contain information about some dimensions.

\item Check the overlap again with {\em lmovl.run}. Standard out now
looks like:
\begin{verbatim}

IB  JB  CL1  CL2 DIST SUMRS 1/d 1/s1 1/s2
=========================================
 1   5  NI   Se  .42  .47    13  22   23
 1  20  NI   E1  .35  .40    13  18   29
-----------------------------------------
 5   1  Se   NI  .42  .47    13  23   22
 5   6  Se   Se  .41  .46    13  23   23
 5  14  Se   E1  .34  .38    13  19   28
 5  15  Se   E1  .38  .38     1   1    2
 5  17  Se   E1  .34  .38    13  19   28
 5  22  Se   E1  .34  .38    13  19   28
 5  23  Se   E1  .34  .38    13  19   28
 5  29  Se   E1  .38  .38     1   1    2
 5  31  Se   E1  .34  .38    13  19   28
 5  34  Se   E1  .34  .38    13  19   28
 5  35  Se   E1  .38  .38     1   1    2
 5  36  Se   E1  .34  .38    13  19   28
-----------------------------------------
13   2  E1   NI  .35  .40    13  29   18
13   6  E1   Se  .34  .38    13  28   19
13   7  E1   Se  .38  .38     1   2    1
13   8  E1   Se  .34  .38    13  28   19
13  12  E1   Se  .34  .38    13  28   19
13  15  E1   E1  .27  .31    12  21   21
13  25  E1   E1  .30  .31     1   2    2
-----------------------------------------
\end{verbatim}

The program found space filling with a
maximum overlap of 13\% which is fine and one can proceed.

\item Complete the CTRL file with the {\em lmctl.run} program, but first
change the token VERBOS= to 50 in category IO to ensure that all
default values are inserted.
\begin{verbatim}
HEADER  Ni4Se8, cubic primitive
VERS    LMASA-47
IO      VERBOS=50 HELP=F WKP=F IACTIV=F
        ERRTOL=2 OUTPUT=CTL ERR=ERR
SYMGRP  NGEN=3 GENGRP=I R3D MX:(.5,.5,0)
        SPCGRP=Pa-3 USESYM=F
STRUC   ALAT=11.2683
        PLAT=1 0 0
             0 1 0
             0 0 1 FIXLAT=T
DIM     NBAS=36 NCLASS=3 NL=3 LDIM=92
        IDIM=232 NSYMOP=24 NKP=11
OPTIONS NSPIN=1 REL=T CCOR=T NONLOC=F
        NRXC=1 NRMIX=2 CORDRD=F
        NITATOM=30 CHARGE=F FATBAND=F AFM=F
        FS=F CARTESIAN=T WRIBAS=F Q=----
CLASS   ATOM=Ni Z=28 R=2.72719158 LMX=2
        CONF=4 4 3 4 IDXDN=1 1 1 IDMOD=0 0 0
        ATOM=Se Z=34 R=2.57576616 LMX=2
        CONF=4 4 4 4 IDXDN=1 1 2 IDMOD=0 0 0
        ATOM=E  Z= 0 R=1.72750257 LMX=2
        CONF=1 2 3 4 IDXDN=1 2 2 IDMOD=0 0 0
SITE    ATOM=Ni POS=0.00000 0.00000 0.0000
        ATOM=Ni POS=0.50000 0.50000 0.0000
        ATOM=Ni POS=0.50000 0.00000 0.5000
        ATOM=Ni POS=0.00000 0.50000 0.5000
        ATOM=Se POS=0.38300 0.11700 -.1170
        ATOM=Se POS=-.38300 -.11700 0.1170
        ATOM=Se POS=-.11700 0.38300 0.1170
        ATOM=Se POS=0.11700 -.38300 -.1170
        ATOM=Se POS=0.11700 -.11700 0.3830
        ATOM=Se POS=-.11700 0.11700 -.3830
        ATOM=Se POS=0.38300 0.38300 0.3830
        ATOM=Se POS=-.38300 -.38300 -.3830
        ATOM=E  POS=0.28510 -.18060 0.0960
        ATOM=E  POS=-.28510 0.18060 -.0960
        ATOM=E  POS=0.09608 0.28510 -.1806
        ATOM=E  POS=-.09608 -.28510 0.1806
        ATOM=E  POS=0.18060 -.09608 -.2851
        ATOM=E  POS=-.18060 0.09608 0.2851
        ATOM=E  POS=0.31939 0.09608 0.2148
        ATOM=E  POS=0.21489 0.31939 0.0960
        ATOM=E  POS=-.31939 -.09608 -.2148
        ATOM=E  POS=-.21489 -.31939 -.0960
        ATOM=E  POS=0.09608 0.21489 0.3193
        ATOM=E  POS=-.09608 -.21489 -.3193
        ATOM=E  POS=0.40391 -.21489 -.1806
        ATOM=E  POS=-.40391 0.21489 0.1806
        ATOM=E  POS=-.18060 0.40391 -.2148
        ATOM=E  POS=0.18060 -.40391 0.2148
        ATOM=E  POS=0.21489 0.18060 -.4039
        ATOM=E  POS=-.21489 -.18060 0.4039
        ATOM=E  POS=0.40391 -.28510 0.3193
        ATOM=E  POS=0.31939 0.40391 -.2851
        ATOM=E  POS=-.40391 0.28510 -.3193
        ATOM=E  POS=-.31939 -.40391 0.2851
        ATOM=E  POS=-.28510 0.31939 0.4039
        ATOM=E  POS=0.28510 -.31939 -.4039
SCALE   SCLWSR=T OMMAX1=.16 .18 .20
        OMMAX2=.40 .45 .50
STR     KAPPA2=0 RMAXS=3.2 NDIMIN=350
        NOCALC=F IALPHA=0
        DOWATS=F DELTR=.1 LMAXW=8
        ATOM=Ni SIGMA=.7 .7 .7
        ATOM=Se SIGMA=.7 .7 .7
        ATOM=E  SIGMA=.7 .7 .7
START   NIT=30 BROY=T WC=-1 NMIX=1 BETA=.5
        FREE=F CNVG=.00001 CNVGET=.00001
        BEGMOM=T CNTROL=T
        EFERMI=-.25 VMTZ=-.75
        ATOM=Ni  P=4.67 4.41 3.86
                 Q=0.6 0.0 0.0
                   0.8 0.0 0.0
                   8.6 0.0 0.0
            enu   =-.543 -.398 -.304
            c     =-.419 0.544 -.281
            sqrdel=0.464 0.580 0.188
            p     =0.054 0.027 2.398
            gamma =0.546 0.240 -.009
        ATOM=Se  P=4.95 4.84 4.17
                 Q=2.0 0.0 0.0
                   3.7 0.0 0.0
                   0.3 0.0 0.0
            enu   =-1.005 -0.388 -0.513
            c     =-1.304 -0.360  1.311
            sqrdel= 0.344  0.387  0.596
            p     = 0.372  0.170  0.024
            gamma = 0.442  0.145  0.150
        ATOM=E   P=1.5 2.5 3.5
                 Q=0 0 0
                   0 0 0
                   0 0 0
            enu   =-1.758 -0.642  1.316
            c     = 0.990  3.503  6.985
            sqrdel= 0.590  0.485  0.400
            p     = 0.006  0.003  0.001
            gamma = 0.358  0.065  0.022
CHARGE  LMTODAT=T ELF=F ADDCOR=F 
        SPINDENS=F
        CHARWIN=F EMIN=-2 EMAX=2
PLOT    ORIGIN=0 0 0
            R1=1 0 0 NDELR1=0
            R2=0 1 0 NDELR2=0
            R3=0 0 1 NDELR3=0
        FORMAT=1
BZ      NKABC=4 4 4 TETRA=T METAL=T
        TOL=.000001
        N=0 W=.005 RANGE=5 NPTS=1001
EWALD   NKDMX=250 AS=2 TOL=.000001
RHOFIT  FIT=F KAPPA2=0 RMAXS=3.5
        ATOM=Ni LMXRHO=2 SIGMA=.7 .7 .7
        ATOM=Se LMXRHO=2 SIGMA=.7 .7 .7
        ATOM=E  LMXRHO=2 SIGMA=.7 .7 .7
SCELL   PLAT=1 0 0
             0 1 0
             0 0 1 EQUIV=T
HARTREE BEGATOM=T LT1=2 LT2=2 LT3=2
DOS     NOPTS=801 EMIN=-2 EMAX=2
SYML    NQ=30 Q1=0.0 0.0 0.0 LAB1=g
              Q2=0.5 0.5 0.0 LAB2=M
        NQ=20 Q1=0.5 0.5 0.0 LAB1=M
              Q2=0.5 0.0 0.0 LAB2=X
        NQ=20 Q1=0.5 0.0 0.0 LAB1=X
              Q2=0.0 0.0 0.0 LAB2=g
        NQ=35 Q1=0.0 0.0 0.0 LAB1=g
              Q2=0.5 0.5 0.5 LAB2=R
FINDES  RMINES=.9 RMAXES=4 NRXYZ=72 72 72
\end{verbatim}

Notice that the Wigner-Sitz radii have been inserted because
SCLWSR=T.

This is a complete CTRL file. The meaning of only a few of these
tokens is necessary in order to run the program, however, a complete
list with a description of each token
is given in Section\ref{categ}.

\item Before calculating
the TB real space structure constants $S^\alpha $ and $\dot
S^\alpha $ for the combined correction term by the
{\em lmstr.run} program, the verbosity is lowered by changing
VERBOS= to 40 in category IO.
$S^\alpha $ only depends on the structure
and the number of partial waves included, while $\dot S$ also depends on
the WS-radii.  If these are changed during the calculations, the
structure constants have to be recalculated.

\item A self-consistent-field (SCF) calculation can now be performed
using the program {\em lm.run}.  If self-consistency is not
achieved within the specified number of iterations (10 in the present
case, the token NIT= in category START), the iterations can
be continued using as start values, the information (potentials,
potential parameters, etc.) in
the atomic files which were generated by {\em lm.run} or from the
potential parameters or moments from the last iteration which the
program wrote into the CTRL file.
If it is decided to start the SCF calculation from scratch, then the
atomic files must be erased. In the present case
the program was converged before the 10th iteration. However, the
default values for the k-mesh in the Brillouin zone are small (4
divisions along each of the primitive reciprocal translation vectors in
the present case (in general it depend on the structure),
see token NKABC= in category BZ).

\item {\em lm.run} is executed again with a larger number of k-points.
This is only necessary for a metal where the Fermi energy and the
Fermi surface must be accurately determined or for insulators if
nice density of states is wanted.
Change token NKABC= to e.g. 16 16 16 (depending on the length of the
reciprocal translation vectors) in category BZ. In category
START, the token BEGMOM= is set to F. This tells the
program to start with a band structure calculation.

\item After the SCF solution to the problem has been found, the results
are analyzed. First,
the band structure along lines in k-space is calculated
using {\em lmbnd.run}. The default values for the lines are given in
category SYML. If orbital projected bands (fat bands) are needed
the token FATBAND= must be changed to T in category OPTIONS. It should
be noticed that if the atomic files are present the potential
parameters are calculated with the enu's in the CTRL file i.e. one can
change the enu's to the range of interest. It should, however, be
noticed that the Fermi energy is not recalculated.
The band structure is plotted by {\em gnubnd.exec} as described in
Section \ref{bandplot}.

\item The atom and orbital projected densities of states (DOS) are
calculated
by {\em lmdos.run}. The default values for the energy window is the
smallest and the largest eigenvalue (which are output from the {\em
lm.run}). These and the number of
mesh points are given in category DOS and they must be changed
appropriately.
To get nice curves also for semiconductors an
iteration with more k-points should be performed by {\em lm.run} as
described above.
The DOS is plotted by {\em gnudos.exec} as described in Section
\ref{dosplot}.

\item The full charge density is calculated by {\em lm.run}
in the charge mode i.e. the token CHARGE= is set to T in category
OPTIONS. If the electron localization function (ELF) is also needed, the
token ELF= is set to T in category CHARGE. The r-mesh on which these
functions are calculated is given in category PLOT which must be
changed. The plots are performed by {\em gnucharge.exec} as described
in Section \ref{charplot} if token FORMAT=1 in category PLOT or with
Data Explorer if FORMAT=3.

\item The Fermi surface data is calculated by {\em lm.run} if the
token FS= is set to T in category OPTIONS. This can be plotted by
{\em gnufs.exec} as described in Section \ref{fsplot} (Note that not
all structures are implemented yet!), or by Data Explorer.

\end{enumerate}

%\input{organ}
\section{Organization of TB-LMTO Programs}
\label{organ}

\subsection{Content of the main directory}
All executable programs are in the main directory.
The TB-LMTO package consists of four classes of programs.

({\em i})
Programs to construct and check the data in the main input file (CTRL):

{\em lminit.run} generates a CTRL file with the structural data.

{\em lmhart.run} generates overlapping potentials from atomic
Hartree potentials or finds MT-radii.

{\em lmovl.run} calculates and displays sphere overlaps.

{\em lmes.run} finds interstitial spheres.

{\em lmctl.run} rewrites the CTRL file according to the parameters in
the original CTRL and atomic files and inserts default values.

{\em lmscell.run} increases the basis for super cell calculations.

{\em lmnghbr.run} produces a table of neighbors for each atom.

{\em lmclean.run} reduces the CTRL file to the essential
crystallographic data.

({\em ii}) Programs to calculate the structure constants
and performe self-consistent calculations:

{\em lmstr.run} generates the structure constants.

{\em lm.run} is the main LMTO program performing self-consistency
iterations.

({\em iii}) Programs to calculate the data to be displayed:

{\em lmbnd.run} generates bands for plotting.

{\em lmdos.run} generates density of states for plotting.

{\em lm.run} generates output for {\em lmdos.run} and charge density
for plotting. (This is the same as under ({\em ii}) above, but run
with different options.)

({\em iv}) Plot programs. The programs with extension {\em exec} will
(1) execute the program with the same name but extension {\em run},
which produces data files for the GNUPLOT program, (2) call the
GNUPLOT program to make the plot, and (3) delete the temporary data
files.

{\em gnudos.exec}: For density of states plot.

{\em gnubnd.exec}: For energy bands plot.

{\em gnucharge.exec}: For charge density plots.

{\em gnufs.exec}: For Fermi surface plots.

All these executable programs are made by the UNIX {\em make} command
which compiles and links the source codes in the subdirectories. All
information about how this should be done is in the {\em makefile},
which is also located in the main directory. The {\em main} programs
for the first three classes of programs above are made from the content
of the file {\em lmall.f} in the main directory. This is done by
{\em make} using the information in {\em makefile}, {\em getmain},
and the program {\em ccomp}.
{\em ccomp} and its source {\em ccomp.c} are in the main directory.
The latter is the only program written in the language {\em C}.
For a description of {\em ccomp.c} see Appendix \ref{prepro}. The reason
why the {\em main} programs are constructed in this way is that when
changes are made, this only has to be done once, namely in
{\em lmall.f}.

\subsubsection{Dimensions/size of the program}
\label{dim}

Since the program is written in FORTRAN77 no true dynamical memory
allocation can be performed.  However, a `semi-dynamical memory
allocation' procedure is implemented.  Apart from small fixed size
arrays all arrays are stored in a work-array called {\em w} in the main
programs.  At the time a new array is needed, say the Hamiltonian, its
size is calculated and space for it is allocated in the work array.
When the array it not needed anymore, the space is freed and other arrays
can be stored in the same place.  This prevents `over-dimensioning'.
The size of the work array, however, depends on the size of the problem
i.e. number of atoms, number of k-points, number of partial waves, etc.
The size of the work array is given by {\em wksize} in the main
programs.  If {\em wksize} is too small for the problem at
hand the program will abort, requesting increase of {\em wksize}.

\subsection{Content of the subdirectories}
After installation of the package each subdirectory contains files of
source codes (.f or .c) and object codes (.o). Each file contains one
main program or one subroutine. These are linked together by the UNIX
{\em make} command to executable programs (.run) using the {\em
makefile}.
\subsubsection{Main programs}

MAIN: All main programs. These are generated by {\em make} as described
above.

MAINA: Some subroutines which have to be linked to the main program
and some routines to calculate the Hartree potential.

\subsubsection{Structure related routines}

TBSTR: Real-space tight-binding structure constants
$S^\alpha $ and $\dot S^\alpha $ for combined-correction term.

FINDES: Programs to find interstitial spheres.

LATTICE: Routines which manipulate the lattice vectors.

HARTREE: Programs to calculate Hartree potentials.

\subsubsection{Atomic programs}

ATOM: Produces potential parameters (pp) and other atomic information
from the moments. Non-relativistic
calculations may be performed by setting REL=F in category OPTIONS.
One can choose between the von Barth-Hedin or the Cheperly-Alder
local density exchange correlation potential. The Langreth-Mehl-Hu and
Perdew-Wang gradient corretions are also implemented.

\subsubsection{Solid state programs}

BNDASA: Calculates energy bands and evaluates the moments.

TETRA: Sets up the k-points and weights for k-space integration using
the tetrahedron method.

SAMPLE: k-space integration by sampling.

MAD: Evaluates the Madelung potential and energy.

DENS: Calculates full charge density and the ELF.

RHONS: Makes a fit to the full charge density.

SYM: Generates all operations in a space group from a set of
generators.  This is used to determine the k-points in the irreducible
part of the Brillouin zone, and to symmetrize the density matrix which
is only calculated in the irreducible BZ.

\subsubsection{Utility subdirectories}

LMIO: Input of the CTRL file using library in IOLIB.

IOLIB: Fortran support for input and file handling.

IOCTRL: Programs to write the CTRL file.

EISP: Matrix diagonalization routines.

LINP: Linear algebra routines.

BLAS: Basic linear algebra routines.

EXTENS: Various utility routines and machine constants.

TEX: This document and the Review of Modern Physics style file
{\em rmp.sty}.

\subsubsection{Plots}

PLOT: Programs to produce the data files used by GNUPLOT. This is the
only subdirectory where the programs are not split into subroutines.

%\input{data}
\section{Data files}
\label{data}

The names of the data files given below should be used,
though the actual name depends on the operating system.
The name of the program which produces the data file is given in
parentheses.

CTRL: An input and documentation file for all programs.
All data in this file are described in detail in Section \ref{categ}.

ATOM-NAME: There is one file assigned to each inequivalent atom.  A
complete file (one generated in the atomic program) contains some
general
information, the moments, potential parameters and the ASA or Hartree
potential and core density within the sphere.  The moments and potential
parameters are most commonly read from this file, but it is possible to
input the moments or the potential parameters
from the CTRL file instead, in which case these
files need not be present at the start of a run. The name ATOM-NAME is
fixed by the string following ATOM= in the CLASS category of the CTRL
file.

The atomic file also has categories delimited by a nonblank character
in the first column.  The categories are: GEN: A table of output
generated by the ATOM program. MOMNTS: The P's and moments Q. PPAR: The
potential parameters.  POT: The potential without the nuclear part
(-2Z/r).
The {\em lmhart.run} stores the Hartree atomic potential here, but after
the run the atomic files are removed in order that they are not used
by mistake as starting potentials in the SCF run.
COR: The core electron density. PHI: $\phi $ and $\dot \phi $.

None of these categories is essential unless the information is
required.  For example, if you begin with a band calculation, it is
essential that the potential parameters be present either here or in the
CTRL file.
If you begin with
the moments, the moments must be present either here or in the CTRL
file.  Like in the CTRL file, the data is formatted.  Having the
potential present, facilitates the iterations towards
self-consistency within the sphere.
({\em lmhart.run and lm.run})

STR: Unformatted file structure constants $S$ and $\dot S$ for the
combined correction. ({\em lmstr.run})

MIXM:
This file contains moments and potential parameters from previous
iterations for the Anderson or Broyden mixing.  This is delited
after the job. ({\em lm.run})

MOMS: Moments belonging to each eigenvalue (eigenvectors are only
summed over the different magnetic states m). This is delited after
the job. ({\em lm.run})

BAND: Eigenvalues and angular momentum weights for density of states
calculation.  ({\em lm.run})

LMDM: Density matrix used for full charge density calculation.
({\em lm.run})

RHO: Full charge density (Including core if the token ADDCOR=T in
category CHARGE). ({\em lm.run})

RHOV: Valence charge density (Only produced if token ADDCOR=F in
category CHARGE). ({\em lm.run})

RHOC: Core charge density (Only produced if token ADDCOR=T in
category CHARGE). ({\em lm.run})

RHOS: Spin density density (Only produced if token SPINDENS=T in
category CHARGE). ({\em lm.run})

RHOF: Fitted charge density. (Only produced if token FIT=T in category
RHOFIT). ({\em lm.run})

BNDS: Energy bands. ({\em lmbnd.run})

EIGN: Eigenvalues and full eigenvectors in the orthogonal representation
- for `fatbands' plot.  Note that this file can be big and in case
`fatbands'
are not needed, set token FATBAND=F in category OPTIONS.
({\em lmbnd.run})

DOS: Atom and orbital projected densities of states.  ({\em lmdos.run})

A number of files with extension .dx are created if the token FORMAT=3
in the category PLOT. These are used in the plots using Data Explorer.

%\input{categ}
\section{Description of the CTRL file, Categories and Tokens}
\label{categ}

Examples of CTRL files are shown in Section \ref{execute}.

The control file is the main input file, which can also serve to
document a calculation.  Data is read from the control file by
categories, one category at a time.  A category begins with a nonblank
character in the first column; it ends with the next occurrence of one.
The name of the category is the string that begins the category; e.g.
the category `STRUC' begins with `STRUC' and ends before `CLASS'.
Data within a category is identified by a token, e.g.  NSPIN=.
Categories can occur in any order, but only the first category of a
given name is used.  Apart from a mild exception described later, the
order of tokens within a category is also irrelevant. Only the 200 first
characters in each line are read and a \# works as an end-of-line
character. See the documentation in IOLIB for a detailed description
of the input procedures.

Several programs modify the CTRL file for instance:
{\em lmhart.run} inserts
MT-radii, {\em lmes.run} inserts interstitial spheres,
{\em lmctl.run} inserts default values for tokens not already
specified in the CTRL file, and {\em lmscell.run} inserts data
from the single cell to the super cell.
The latter two changes the sphere radii
to space filling spheres if the token SCLWSR=T in category SCALE.
Furthermore, if {\em lm.run} ran successfully then the new potential
parameters, the moments, the Fermi energy and the muffin-tin zero is
written in the CTRL file. This means that the BAND file need not be
present when running {\em lmbnd.run} (previously the Fermi energy was
read from this file) and the atomic files need not be present when
continuing the iterations (although it is somewhat faster to continue
with the atomic files because the potentials are present).

A complete list of categories and an explanation of each token is
given below.

Each subsection contains the name of the category and each
`subsubsection' contains the name of the token.

\subsection{Category HEADER (optional)}
Provides space to describe the contents of the
control file.

\subsection{Category VERS}

\subsubsection{Token LMASA- of cast integer}
Is the version number to ensure compatibility of the CTRL
file with the executing program. The program will send a warning
message if the version number is incorrect, but it will not stop.

\subsection{Category IO (optional)}

\subsubsection{Token HELP= of cast logical (optional)}
T: print a list of tokens and data appropriate for this program, without
reading anything.

\subsubsection{Token VERBOS= of cast integer (optional)}
Is an integer that determines the verbosity of output a program
sends to standard out.  The output is roughly as follows:

0: nearly nothing is printed 10: very terse 20: terse 30: normal 40:
normal 50: verbose 100: low-level debugging 110: intermediate level
debugging 120: high-level debugging.

\subsubsection{Token WKP= of cast logical (optional)}
Is a switch that turns on the `debug' mode in the dynamic
memory routines.  An effective heap is declared as a single integer
array in the main program.  Pieces are apportioned dynamically by calls
to routine defdr.f and others.  They can be `released' to be reused.
You can watch the memory grow and shrink if this is turned on.

\subsubsection{Token IACTIV= of cast logical (optional)}
Is a switch that turns on the `interactive' mode.  When
this mode is on, you have an option to abort program execution, change
the verbosity, turn on the work array debug switch WKP= , or sometimes
to change a value of a single variable that may be passed to query.
There is also the option to turn off the interactive mode.  At the
prompt, enter either a simple carriage return, or one of the following:
`Vnnn', where nnn is the new verbosity; `W' to toggle printing of work
array; `I' to turn off interactive prompt; `A' to abort program
execution or, in the case when a variable is passed, you may also enter
`Snnn', where nnn is a number (or T or F for logical variable); to
change the value of that variable.

\subsubsection{Token ERRTOL= of cast integer (optional)}
An error or inconsistency discovered by the program is assigned an
integer error code from 0 to 4. If the error code is larger or
equal to the value of ERRTOL, the program will stop. The meaning of the
error codes are: 0; information, 1; warning, 2; error (this is the
default), 3; severe error, meaning that the results cannot be trusted,
4; fatal error, the program will always stop. For ERRTOL $>$ 4, ERRTOL
is set to 4. For ERRTOL=--1 all errors will be ignored and the program
will continue.

\subsubsection{Token OUTPUT= of cast character (optional)}
A file name to which standard out is routed. OUTPUT=* means standard
out.

\subsubsection{Token ERR= of cast character (optional)}
A file name to which standard error is routed.
ERR=* means standard error.

\subsection{Category DIM}
In this category some parameters related to the dimension of the
problem is given. These are output from the programs and not needed to
run the programs.

\subsubsection{Token NBAS= of cast integer (optional)}
Is the number of atoms in the basis. This is not used as an input
to the calculation.

\subsubsection{Token NCLASS= of cast integer (optional)}
Is the number of classes (inequivalent atoms not related by symmetry).
This is not used as an input to the calculation.

\subsubsection{Token NL= of cast integer}
Maximum number of $\ell $ values on any sphere. This need not be given.

\subsubsection{Token LDIM= of cast integer}
The number of low partial waves.

\subsubsection{Token IDIM= of cast integer}
The number of intermediat (down-folded) partial waves.

\subsubsection{Token NSYMOP= of cast integer}
The number of symmetry operations in the space group.

\subsubsection{Token NKP= of cast integer}
The number of irreducible k-points.

\subsection{Category STRUC}

\subsubsection{Token ALAT= of cast double}
Is a scaling factor, in atomic units (Bohr radii), of the primitive
translation and basis vectors.

\subsubsection{Token PLAT= of cast double and length 9}
Are primitive translation vectors, ${\bf t}_1, {\bf t}_2, {\bf t}_3$,
in units of ALAT and in Cartesian coordinates. The order is as follows:
$t_{1x} t_{1y} t_{1z} t_{2x} t_{2y} t_{2z} t_{3x} t_{3y} t_{3z}$.

\subsubsection{Token FIXLAT= of cast logical}
The program may change the lattice. If this is not wanted choose
FIXLAT=T. (FIXLAT=T is default.)

\subsection{Category OPTIONS (optional)}
Is self explanatory.

\subsubsection{Token NSPIN= of cast integer (optional)}
Must be 1 for non-spin-polarized or 2 for spin-polarized
calculations.

\subsubsection{Token REL= of cast logical (optional)}
T: The scalar relativistic wave equation is solved.
F: The non-relativistic Schr\"odinger equation is solved.

\subsubsection{Token CCOR= of cast logical (optional)}
Turns on the combined correction.  Program {\em lmstr.run} always
calculates the $\dot S^{\alpha }$.  They are stored in a file SDOT for
use by program {\em lm.run} and {\em lmbnd.run},
which will include the combined correction
into the LMTO Hamiltonian and overlap matrices
when this switch is set to T.

\subsubsection{Token NONLOC= of cast logical (optional)}
T: Turns on one of the non-local exchange correlations specified by the
token NRXC.
\subsubsection{Token NRXC= of cast logical (optional)}
For NRXC=1 the von Barth-Hedin local exchange correlation potential is
used. If the token NONLOC=T the Langreth-Mehl-Hu non-local exchange
correlation is added.
For NRXC=2 the Vosko-Wilk-Nusair local exchange correlation potential is
used. If the token NONLOC=T the Perdew-Wang non-local exchange
correlation is added.

\subsubsection{Token NRMIX= of cast integer (optional)}
Determines the number of previous iterations in the charge density to 
mix when making a sphere self-consistent.  Anderson mixing is
used to mix the charge density to facilitate convergence.  Practical
experience has shown that NRMIX=2 works well (this is the default), and
normally you should not need to worry about this parameter.  It may
occasionally happen that the sphere program will not converge with
Anderson mixing, in which case you should set NRMIX to 0.

\subsubsection{Token Q= of cast character (optional)}
BAND, MAD, ATOM, SHOW make the program stop after the
band structure calculation, the Madelung calculation etc.

\subsubsection{Token CORDRD= of cast logical (optional)}
T: Reads the core density from the atomic file.  The core
density is not calculated inside the atomic loop but kept frozen.
(Default is F.)

\subsubsection{Token CARTESIAN= of cast logical (optional)}
T: The position of the atoms is given in cartesian coordinates and in
units of ALAT.
F: The position of the atoms is given in units of the translation
vectors (length and direction) a, b, c. For non-symmorphic groups, also
the non-primitive translation vectors associated with the space group
operations are given in these units.

\subsubsection{Token NITATOM= of cast integer (optional)}
Number of atomic iterations.  To make any atom self-consistent
NITATOM=80 suffices.  However, there is no need for
fully self-consistent atoms in the beginning of the band structure
loops since the moments are not fully self-consistent. The default is 30

\subsubsection{Token CHARGE= of cast logical (optional)}
T: Full charge density is calculated.  Other Tokens are
automatically set to the right values.  (Only active for {\em lm.run}.)

\subsubsection{Token FATBAND= of cast logical (optional)}
T: a file (EIGN) is produced by {\em lmbnd.run} containing the
eigenvalues and eigenvectors. This is used by the plot program to
give the energy bands a width proportional to specified orbital
characters. The default is F.

\subsubsection{Token AFM= of cast logical (optional)}
T: Anti ferromagnetic calculation.  I.e. the electronic
structure is only calculated for one spin direction.  It should be
noticed that the atoms in
the two spin sublattices must occur in exactly the same order in the
CTRL file as explained in Section \ref{spin}.

\subsubsection{Token FS= of cast logical (optional)}
T: The program {\em lm.run} produces a file LMFS, which can
be used to plot the Fermi surface.

\subsubsection{Token OVLCOR= of cast logical (optional)}
T: A correction for the sphere overlap is added to the hamiltonian
and overlap matrices. This is not well optimized, therefore, it is
recommended to use the default value F.

\subsubsection{Token WRIBAS= of cast logical (optional)}
T: The {\em lmctl.run} program will create a file cstruc.dx with
all atomic positions within the volume defined in category PLOT.
This can be used by DATA EXPLORER to make a plot of the structure.
If {\em lmctl.run} is executed interactively (token IACTIV=T in
category IO) the colors and the `sticks' connecting the atoms
can be chosen, otherwise they are chosen by the program.

\subsubsection{Token SEWALD= of cast logical (optional)}
T: The structure constants are calculated by the Ewald method. (Default
is F.) This could be usefull if it is difficult to converge the
structure constants by the real space method.

\subsection{Category CLASS}
Contains information relevant to parameters inside atomic
spheres for each inequivalent atom.

\subsubsection{Token ATOM= of cast character}
A one-- (or more) character string naming the atom of
that class.  This string names the disk file -- at least for some
operating systems -- that will hold important information about that
atom (potential parameters, moments, potential etc.),
and is used elsewhere in the input (see categories SITE
and START below) to identify a particular atom.  Following the token
ATOM=, several tokens are sought:

\subsubsection{Token NEWNAM= of cast logical}
T: The program is allowed to change the names of the atoms.

\subsubsection{Token Z= of cast double}
The nuclear charge of the atom. For interstitial spheres Z=0.

\subsubsection{Token R= of cast double (optional)}
The sphere radius, in atomic units (Bohr radii).

\subsubsection{Token LMX= of cast integer (optional)}
The maximum $\ell $ quantum number inside the sphere.

\subsubsection{Token IDXDN= of cast integer and length 4 (optional)}
Can take the value 0, 1, 2, or 3, for each $\ell $. This
determines if the partial wave should be treated as low (1),
intermediate (2) (i.e. it will be downfolded), or high (3) (i.e. it will
be thrown away). If IDXDN=0;
then the program decides if the partial wave should
be treated as low, intermediate, or high.

\subsubsection{Token CONF= of cast integer and length 4 (optional)}
Configuration i.e. a principal quantum number.
This determines which states will be treated
as core states.  For each $\ell $ all states with principal quantum
number smaller than that given by CONF will be treated as core states.

\subsubsection{Token IDMOD= of cast integer and length 4 (optional)}
Is a set of integers, one for each $\ell $-channel.  IDMOD
controls how the potential parameter E$_\nu $ changes from one
iteration to the next in a self-consistency cycle.

IDMOD=0 lets the E$_\nu $'s
shift to the center of gravity of the occupied part
of the band (this is the default);

IDMOD=1 essentially freezes the
logarithmic derivative of the wave function phi at the sphere radius;

IDMOD=2 freezes the E$_\nu $'s from
one iteration to the next.

Because data must be read for each atom, tokens are repeated (once for
each class).  For that reason, there is some restriction as to the order
of tokens.  Tokens Z=, R=, R/W=, LMX= and CONF= must follow the token
ATOM= they are associated with and precede the next token ATOM=.

\subsection{Category SITE}
Holds site information.

\subsubsection{Token ATOM= of cast character}
Is needed to identify which atom in category CLASS that site
belongs to.  Similarly to category CLASS, one set of tokens is read for
each site in the basis, and there is a similar restriction on the order
of tokens.

\subsubsection{Token POS= of cast double and length 3}
Is a basis vector, in units of ALAT and in
cartesian coordinates. The basis could alternatively be given in units
of the translation vectors {\bf a}, {\bf b}, and {\bf c} (see token X=).
Some of the programs require the most compact basis position i.e.
positions closest to the origin. These new positions are calculated
by the programs and written back into the CTRL file.

\subsubsection{Token X= of cast double and length 3}
Is a basis vector, in units of the translation vectors {\bf a}, 
{\bf b}, {\bf c}. The basis could alternatively be given in units of
ALAT and in cartesian coordinates (see token POS=).
This is only read if the token CARTESIAN=F in
category OPTIONS (see this and the description to the token POS=).

\subsection{Category SYMGRP (optional)}
Provides information to confine integrations to a
smaller part of the Brillouin zone.  One enters a small number of space
group operations; subroutines in the subdirectory SYM
will generate all
inequivalent products of these operations.  For example, the three
operations R4X, MX and R3D are sufficient to generate all 48 elements of
the cubic octahedron symmetry.  Symbols have the form O:(nx,ny,nz) where
O is one of M, I or Rj for mirror, inversion and j-fold rotation; and
nx,ny,nz are a triplet of indices specifying the axis of operation.  You
may use X, Y, Z or D as shorthand for (1,0,0), (0,1,0), (0,0,1),
(1,1,1).  You may also enter products, such as I*R4X.  Since these
operations restrict numerical integration to a portion of the BZ, it is
important that the class definitions in category CLASS are compatible
with these operations.  If these operations must assume two atoms to be
equivalent but are not so defined in CLASS, the electron density in
those two atoms may not be equivalent.  For non-symmorphic space groups
the translational part of the group operation must also be specified
(this is only used for calculating the density matrix, i.e. for charge
density calculation).  This is done by adding after the rotational part
:(rx,ry,rz).  I. e. for the NiSe2 example above MZ:(0.5,0,0.5). If the
token CARTESIAN=F in category OPTIONS the translation is given in terme
of the chemists a, b, and c and the rotation part is separated from the
translation part by ::. I. e. MZ::(0.5,0,0.5).

\subsubsection{Token NGEN= of cast integer (optional)}
The number of generators that follows.

\subsubsection{Token GENGRP= of cast integer (optional)}
The names of the generators separated by a blank.

\subsubsection{Token USESYM= of cast logical (optional)}
T: The space group generated from the generators is used.
This may be useful if the calculation has to be performed for a lower
symmetry than the actual symmetry of the solid. F: The space group of
the solid is found by the program
and used in the calculation.
Furthermore, if some atoms are missing in
category SITE they will be added to the basis.  I.e. only one of the
equivalent atoms must be supplied in SITE.
After the basis has been completed it is recommended to use USESYM=F.

\subsubsection{Token SPCGRP= of cast character (optional)}
Is the name of the space group. See Ref. 8.

\subsubsection{Token IORIGIN= of cast integer (optional)}
Can take the values 1 or 2 depending on the choice of origin. The same
convension as in Ref. 8 is used. This is used for generation the CTRL
file by {\em lminit.run}.

\subsection{Category SCALE}
Contains information about scaling the sphere radii.

\subsubsection{Token SCLWSR= of cast logical (optional)}
T: Scale the sphere radii. F: Do not scale.

\subsubsection{Token FACVOL= of cast double (optional)}
Scale the sphere volumes to FACVOL$\times $ the unit cell volume.
Default is FACVOL=1.

\subsubsection{Token OMMAX1= of cast double and length 3 (optional)}
Scale to an overlap (s1+s2--d)/d $<$ OMMAX1, keeping s1/s2 ratio.
s1 and s2 are the two sphere radii and d is the distance between the
centers of the spheres. The three numbers are for atom-atom,
atom-interstitial, and interstitial-interstitial overlap, respectively.
The default values are: 0.16, 0.18, and 0.20.

\subsubsection{Token OMMAX2= of cast double and length 3 (optional)}
Scale to an overlap (s1+s2--d)/s $<$ OMMAX2, keeping s1/s2 ratio.
s1 and s2 are the two sphere radii, s is the smallest of these, and d
is the distance between the centers of the spheres.
The three numbers are for atom-atom,
atom-interstitial, and interstitial-interstitial overlap, respectively.
The default values are: 0.40, 0.45, and 0.50.

\subsubsection{Token GAMMA= of cast double (optional)}
Linear scaling of radii to space filling volume. s(new) = a(s(old)+b),
where ab=GAMMA(a-1) $\times $ the average Wigner-Seitz radius.

\subsection{Category STR}
Contains information for the structure constant program.
Real space tight-binding structure constants are generated for a cluster
of atoms inside RMAXS$\times $W around each basis atom by `inversion'
using Cholesky decomposition. See the explanation in the subdirectory
TBSTR.

%\subsubsection{Token NKAP= of cast integer (optional)}
%number of kappa values.  The program is prepared for two
%kappas but only one kappa is fully implemented.  Default is 1.

\subsubsection{Token KAPPA2= of cast double (optional)}
Kappa square.  The default is zero, but the program works
for positive as well as negative kappas.  Alternatively KW**2 may be
used as input.

\subsubsection{Token KW**2= of cast double (optional)}
(Kappa$\times $average Wigner-Seitz radius) squared.

\subsubsection{Token IALPHA= of cast integer (optional)}
IALPHA=0: $\alpha _{l,R}=
j_l(kappa\times SIGMA\times s_R) /
n_l(kappa\times SIGMA\times s_R) $
is used. Where SIGMA is a token.

IALPHA=1: $\alpha $'s are calculated in the program.
$\alpha $ and $\dot \alpha $ depend on kappa.
The canonical $\alpha $'s only applies for kappa=0.  Optimal $\alpha $'s
(leading to screening for non-zero kappa's) have been calculated
as a function of kappa and a fit formula has been found.  This is used
to calculate the $\alpha $'s.

IALPHA=2: $\alpha $ (token ALPHA) and $\dot \alpha $ (token ADOT)
must be supplied in the CTRL file.

\subsubsection{Token ALPHA= of cast double (optional)}
A list of $\alpha $ values which are read as follows:
\begin{verbatim}
STR ATOM=i ALPHA=al(0,i) al(1,i) ... al(lmx,i)
           ADOT= ad(0,i) ad(1,i) ... ad(lmx,i)
    ATOM=j ALPHA=al(0,j) al(1,j) ... al(lmx,j)
           ADOT= ad(0,j) ad(1,j) ... ad(lmx,j)
    ...
\end{verbatim}
Where the al's and ad's are numbers (one for each $\ell $ and
atom) ALPHA and ADOT are tokens.

\subsubsection{Token ADOT= of cast double (optional)}
ADOT is $\dot \alpha $. For format of input, see the token ALPHA.
Only used if CCOR=T.

\subsubsection{Token SIGMA= of cast double (optional)}
Screening core radii used to make $\alpha $'s if the token IALPHA=0.
They are read as follows:
\begin{verbatim}
STR ATOM=i SIGMA=si(0,i) ... si(lmx,i)
    ATOM=j SIGMA=si(0,j) ... si(lmx,j)
    ...
\end{verbatim}
Where the si's are numbers, one for each $\ell $ and atom.

\subsubsection{Token DOWATS= of cast logical (optional)}
T: A `Watson sphere' with a radius enclosing all atomic
spheres in the cluster plus DELTR (Token DELTR below) $\times $ W (=the
average Wigner-Seitz radius) is placed around each cluster of atoms.  On
this sphere the envelope functions expanded in spherical harmonics is
zero for all $l$'s below LMAXW (Token LMAXW below).

\subsubsection{Token DELTR= of cast double (optional)}
See Token DOWATS above.

\subsubsection{Token LMAXW= of cast integer (optional)}
See Token DOWATS above.

\subsubsection{Token RMAXS= of cast double (optional)}
Is the maximum sphere radius (in units of the average WS)
in which neighbors will be included in the generation of structure
constants.  This takes a default value and is not required as input.
For `strange' structures one should calculate the structure constants
for several RMAXS and check if the structure constants are converged.

\subsubsection{Token NDIMIN= of cast integer (optional)}
For positive values of
NDIMIN, the number of sites in the cluster around each site is
determined as the total number of orbitals in the cluster $>$ NDIMIN.
The number of orbitals on each site is weighted by WSR(tail)/WSR(head).
I. e. NDIMIN $<$ Sum(i) (number of orbitals at site i)$\times $s(i)/s(j)
for the cluster around site j. 
This is not done if NDIMIN=0. The default value is 350.

\subsubsection{Token ITRANS= of cast integer (optional)}
This token is only effective for the {\em lmstr.run} program. It can
take three values: 0, the usual TB-structure constants; 1 or 2,
new structure constants. Default is 0.

\subsubsection{Token DONALP= of cast logical (optional)}
T: Plot a screened envelope function: $\| N\rangle + \epsilon
\| \dot N\rangle$. See Ref. 10. The following tokens:
LDN, MDN, JBASDN and EPS must be supplied.

\subsubsection{Token LDN= of cast integer (optional)}
See token DONALP. LDN is the $\ell $-value of the function.

\subsubsection{Token MDN= of cast integer (optional)}
See token DONALP. MDN is the $m$-value of the function.

\subsubsection{Token JBASDN= of cast integer (optional)}
See token DONALP. JBASDN is the atom number on which the function
is centered.

\subsubsection{Token EPS= of cast double (optional)}
See token DONALP. EPS is the $\epsilon$-value of the function.

\subsubsection{Token NOCALC= of cast logical (optional)}
Is available if it is desired to read the structure
constants from the file without recalculating them.  This is mostly
useful if you would like to look at them (by turning the
verbosity above 40) without recalculating them.

\subsection{Category BZ}
Holds information concerning the numerical integration of
energy bands, etc. over the Brillouin Zone.  The LMTO programs are not
tied to any particular method, so the desired method must be specified
by a token. In all
methods, the BZ is split into a mesh of parallelepipeds.

\subsubsection{Token NKABC= of cast integer and length 3}
Is the number of divisions of the three primitive
reciprocal translation vectors.  This is preferable to the alternative
token MAXKP below, since the divisions
can be chosen so that the step length is about the same along the three
vectors.

\subsubsection{Token MAXKP= of cast integer}
Is the maximum number of k-points in the entire Brillouin zone.

\subsubsection{Token TETRA= of cast logical (optional)}
T: k-space integration by the tetrahedron method. F:
integration by sampling.  (Don't use F!!!  O.J.)

\subsubsection{Token METAL= of cast logical (optional)}
T: for metals. F: for insulators.  If not known or for the
first iterations choose T.

\subsubsection{Tokens N=, W=, RANGE=, and NPTS=}
Parameters used in k-space
integration by sampling.

\subsubsection{Token TOL= of cast double (optional)}
Is the accuracy with which the Fermi level will be determined.

\subsection{Category EWALD (optional)}
Holds information controlling the Madelung sums.  The
defaults are almost always adequate; for a detailed description the
reader is referred to the documentation on the Madelung sums in the
program in the subdirectory MAD.
For `strange' structures one should check the convergence by
decreasing the token TOL= (see below).

\subsubsection{Token AS= of cast double (optional)}
Controls the relative number of lattice vectors in real and
reciprocal space.

\subsubsection{Token NKDMX= of cast integer (optional)}
Maximum number of terms in the Ewald sum (used for both
real space and reciprocal space sums)

\subsubsection{Token TOL= of cast double (optional)}
Is the error criterion for the Ewald sums.

\subsection{Category DOS}
Holds information about the energy mesh in the density of
states calculations.

\subsubsection{Token NOPTS= of cast integer}
Number of energy mesh points.

\subsubsection{Token EMIN= of cast double}
First energy mesh point.

\subsubsection{Token EMAX= of cast double}
Last energy mesh point.

\subsection{Category SYML}
Holds information about lines in k-space along which the
band structure should be calculated and plotted.

\subsubsection{Token NQ= of cast integer}
The number of points along the line.

\subsubsection{Tokens Q1, Q2= of cast double and length 3}
The first point and last point on
the line in Cartesian coordinates and in units of
2$\pi $/ALAT, where ALAT is the lattice constant given in the CTRL
file.

\subsubsection{Tokens LAB1, LAB2= of cast character}
The labels of the first and last point on the line. Because of
limitations in the present version of gnuplot the $\Gamma $ point
should be given the label $G$.

\subsection{Category START (optional)}
Controls the flow of execution and the mixing scheme used
in the iterations towards self-consistency in the
{\em lm.run} program.  Either
the modified Broyden mixing or Anderson mixing scheme can be used to get
an optimal input vector of P and Q for the next iteration to
self-consistency.  For more information, see subroutines {\em mixpq.f},
{\em mixpqa.f} and {\em mixpqb.f} in the subdirectory MAINA.

\subsubsection{Token NIT= of cast integer (optional)}
Is the number of iterations {\em lm.run} executes before quitting
(unless self-consistency is achieved earlier).

\subsubsection{Token BROY= of cast logical (optional)}
T: Modified Broyden mixing. F: Anderson or linear mixing.

\subsubsection{Token WC= of cast double (optional)}
Specifies how strong previous iteration steps are weighted in
the modified Broyden scheme. For WC large, the last iteration step is
weighted most.  When starting with bad moments it is useful to cancel
the stored Jacobi matrix after some iterations to let the calculation
forget its history (delete file MIXM).
You can also choose an
arbitrary negative value for WC.  In this case, WC is adjusted
internally according to the inverse root mean square difference of the
components of the mixed vector.

\subsubsection{Token NMIX= of cast integer (optional)}
Is the number of previous iterations mixed in with the present
one using the Anderson scheme. Use NMIX=0 for linear mixing and the
the token BETA= the feedback parameter.

\subsubsection{Token BETA= of cast double (optional)}
Is the proportion of the new iteration (as obtained from the
Anderson mixing) admixed with the (1-BETA) of the old iteration. When
difficulties with convergence occur BETA should be reduced.

\subsubsection{Token CNVG= of cast double (optional)}
Is the tolerance in the RMS change in the zeroth moments before
self-consistency is achieved.

\subsubsection{Token CNVGET= of cast double (optional)}
Is the tolerance in the total energy before self-consistency
is achieved.

\subsubsection{Token FREE= of cast logical (optional)}
Is intended to facilitate a self-consistent free-atom
calculation.  When FREE is true, the program uses R=30 for the sphere
radius rather than whatever R is passed to it; the boundary
conditions at R are taken to be value=slope=0 (R=30 should be
large enough that these boundary conditions are sufficiently close to
that of a free atom.); subroutine {\em atscpp.f} in subdirectory ATOM
does not calculate potential
parameters or save anything to disk; and program
{\em lm.run} terminates after all
the atoms have been calculated.  Token FREE= T, and running {\em
lmbnd.run}, free-electron bands are produced.  In this case the
structure constants are not used.

\subsubsection{Token BEGMOM= of cast logical (optional)}
T: Causes program {\em lm.run} to begin with moments from
which potential parameters are generated. F: the potential
parameters are used and the program proceeds directly to the band
calculation.

\subsubsection{Token CNTROL= of cast logical (optional)}
T: Read and use moments or potential parameters
supplied in the CTRL file.
F: The data following CNTROL are not used.  Both the
`continuous principal quantum numbers' P (as described in the section
`Iterations Towards Self-Consistency') and the Q's (the moments) must be
present for a given atom as displayed in the example; however, it is not
necessary to have an input for each atom.  Moments are read in Q0, Q1,
Q2, once for each $\ell $ channel,
and then once for each spin, if there are
two spins.  In the case ATOM=NI in the above example, 0.6 electrons
is put in the s orbital, 0.8 in the p and 8.6 in the d.

\subsubsection{Token EFERMI= of cast double precision (optional)}
The Fermi energy. This is an output from the program. If it is not
calculated a default value of -0.25Ry is inserted.

\subsubsection{Token VMTZ= of cast double precision (optional)}
The muffin-tin zero. This is an output from the program. If it is not
calculated a default value of -0.75Ry is inserted.

\subsection{Category HARTREE}
Gives information about the calculation of the overlapping
Hartree potential.

\subsubsection{Tokens LT1, LT2, LT3= of cast integer}
The solid Hartree potential is a superposition of
atomic Hartree potentials.  Atoms in unit cells translated from --LT1
to +LT1 (and equivalently for LT2 and LT3) are taken into account.

\subsubsection{Token BEGATOM= of cast logical (optional)}
T: Calculation of the SCF potential for each atom (Notice that the
token FREE is effective) and keep the Hartree part.
F: The potentials are read from the atom files.  After this an
overlapping Hartree potential for the solid is constructed and stored
in the file POT.

\subsection{Category PLOT}
Gives the mesh on which the full charge density, the ELF,
or the Hartree potential has to be calculated. The mesh can be given
by the three vectors R1, R2, and R3 or on a sphere in which case the
radius (RMIN and RMAX), the angle theta (TMIN and TMAX in the range
from 0 to 180 degrees), and the angle phi
(PMIN and PMAX in the range from 0 to 360 degrees) has to be given.

\subsubsection{Token ORIGIN= of cast double and length 3}
Origin of the mesh in Cartesian coordinates and in units of ALAT.

\subsubsection{Tokens R1, R2, and R3= of cast double and length 3}
Are three vectors spanning a parallelepiped in
which the charge density is calculated.  Units of ALAT.

\subsubsection{Tokens RMIN and RMAX= of cast double}
Radii in a spherical coordinate system.

\subsubsection{Tokens TMIN and TMAX= of cast double}
Theta angles in a spherical coordinate system.

\subsubsection{Tokens PMIN and PMAX= of cast double}
PHI angles in a spherical coordinate system.

\subsubsection{Tokens NDELR1, NDELR2, and NDELR3= of cast integer }
Number of mesh points along the R1, R2, and R3 vectors respectively, or
from RMIN to RMAX, from TMIN to TMAX, and PMIN to PMAX respectibely.
If NDELR3=0
the charge density is evaluated in the plane spanned by R1 and R2.
If NDELR2= is also 0
the charge density is evaluated along the line R1.
For spherical coordinates NDELR1=0 gives plots on a sphere.

\subsubsection{Token FORMAT= of cast integer (optional)}
Selects the format of the output file. FORMAT=1, the format is
appropriate for the GNUPLOT. =3, the format appropriate for DATA
EXPLORER. =2, the format for 2D plots: x, y, f(x,y) and for 3D plots
x, y, z, f(x,y,z), where x, y, and z are the x-, y-, and z-coordinate
of the mesh point and f is the value of the function at this point.
There is one mesh point per line.
This format is appropriate for other plot packages.

\subsection{Category CHARGE}
Information about the full charge or spin density calculation.

\subsubsection{Token LMTODAT= of cast logical}
T: for the first plane in which the charge density has to
be evaluated.  For subsequent planes some data are already present (i.e.
LMDM) and need not be recalculated, therefore LMTODAT= F.
Default is T.

\subsubsection{Token CHARWIN= of cast logical (optional)}
T: Only states in the energy window from EMIN to EMAX
are included.  F: All occupied states are included (default).

\subsubsection{Token EMIN= of cast double}
For CHARWIN=T, EMIN is the minimum energy in the energy window.

\subsubsection{Token EMAX= of cast double}
For CHARWIN=T, EMAX is the maximum energy in the energy window.

\subsubsection{Token ELF= of cast logical}
T: in addition to the charge density, the electron
localization function is calculated.

\subsubsection{Token ADDCOR= of cast logical}
T: The core charge density is added to the valence charge density. F:
The core is not added, which is the default.

\subsubsection{Token SPINDENS= of cast logical}
T: The spin density is also calculated.

\subsection{Category FINDES}
Information about the search for interstitial spheres. The program
{\em lmes.run} first checks if the atomic muffin-tin radii found by
{\em lmhart.run} can be scaled to fill space without violating the
overlap rules in category SCALE. If this is the case then the program
stops. If it is not the case the program searches for interstitial
spheres (ES). For each ES found with radius larger than the token RMINES
(see this) the space filling is checked (scaling to space filling and
checking overlaps). If space filling is reached the program stops,
otherwise more ES are searched for. The program will eventually stop 
either because space filling has been reached or no more ES can be
found with radii larger than RMINES. In the latter case RMINES has to
reduced.

\subsubsection{Token RMINES= of cast double (optional)}
No sphere with radius smaller than RMINES (in a.u.) should be found.
The default value is 1.25 a.u.

\subsubsection{Token RMAXES= of cast double (optional)}
No sphere with radius larger than RMAXES (in a.u.) should be found.
The default value is 4.5 a.u.

\subsubsection{Token MAXPT= of cast integer (optional)}
A real space mesh is used in the search for
interstitial spheres. MAXPT is approximately the total number of points
in the unit cell. Alternatively the token NRXYZ= below can be used.

\subsubsection{Token NRXYZ= of cast integer and length 3 (optional)}
A real space mesh is used in the search for
interstitial spheres. The three numbers NRXYZ are the divisions
of the three translation vectors.

\subsection{Category SCELL}
Information for supercell calculations.

\subsubsection{Token PLAT= of cast double and length 9}
Primitive translation vectors for the supercell. This is equivalent
to the token PLAT in category STRUC and should be compatible with
this i.e. the supercell translations should be a multiplum of those
in STRUC.

\subsubsection{Token EQUIV= of cast logical (optional)}
T: The atomic data in category CLASS and SITE for the supercell will
be completed from the single cell data.

\subsection{Category RHOFIT}
This category holds information about fitting the charge density. It is
not recommended to use it!

\subsubsection{Token FIT= of cast logical}
T: Fit the charge density.

\subsubsection{Token KAPPA2= of cast double}
Energy of the fitting functions.

\subsubsection{Token KW**2= of cast double}
EKAP (see above) times the average Wigner-Seitz radius squared.

\subsubsection{Token RMAXS= of cast double}
Size of cluster used for calculating the structure constants. Same as
in category STR.

\subsubsection{Token SIGMA= of cast double}
Fitting radius. This is read in the same way as in category STR.
\begin{verbatim}
RHOFIT ATOM=i LMXRHO=2 SIGMA=s(0,i) ... s(lmx,i)
       ATOM=j LMXRHO=2 SIGMA=s(0,j) ... s(lmx,j)
       ...
\end{verbatim}
Where the s's are numbers. The value of LMXRHO is only an example.

\subsubsection{Token LMXRHO= of cast double}
Maximum $\ell $ as each site. See token SIGMA= how it is read.


%\input{iter}
\section{Iterations Towards Self-Consistency }
\label{iter}
Just as the LMTO method breaks naturally into an atomic part and a
`solid' part, so do the programs.  In general the `solid' part requires
potential parameters and structure constants as its input, from which it
generates bands, energy moments, densities-of-states, etc.  The `atomic'
part takes moments as its input and calculates potential parameters from
it.  The atomic part requires very little information beyond the moments
and E$_\nu $'s or boundary conditions to completely specify the
electronic structure within an atomic sphere.

An LMTO calculation is self-consistent when the atomic part produces,
from moments generated by the solid part, once again the same potential
parameters that the solid part used to generate the moments in the first
place.  The self-consistency works by alternating between the solid part
and atomic part, generating moments, then potential parameters, then
moments again until the process converges.  The program can be started
either with the solid part, specifying potential parameters, or with the
atomic part, specifying the moments.

Because the method is a linear one, only three functions can carry
charge inside a sphere ($\phi ^2, \phi \dot \phi, \dot \phi ^2$) and
therefore the properties of a sphere, within the approximations of the
linear method are completely determined by the first three moments, the
atomic number and the E$_\nu $'s or boundary conditions at the surface
of the sphere.  In some sense these numbers are `fundamental' to a
sphere; the atomic program will generate a self-consistent potential for
a specified set of moments and E$_\nu $'s and generate potential
parameters from it.

From the point of view of the bands, the Hamiltonian is completely
specified by the potential parameters (and the structure constants).
These are fundamental to the band program; it will generate moments from
the eigenvectors of the Hamiltonian.  Full self-consistency is achieved
when the `input moments' coincide with the `output moments', or
equivalently when the input pp's coincide with the output pp's.

The ASA program can start equally as well from either potential
parameters or moments, though it is generally easier to start from the
moments, if no prior information is available.  This is because one can
usually begin with a very simple starting guess (choosing the zeroth
moment to be the charge of the free atom, the first and second to be
zero) that is usually good enough to iterate towards self-consistency.

If potential parameters are available, you may choose to
begin directly with a band calculation and need not worry about the
moments. If you wish to make a self-consistent
calculation, you must also supply the principal quantum numbers for each
$\ell $ channel in the so-called P number described in the following
paragraphs.

To make a sphere self-consistent one needs the moments and also to
specify the boundary condition on the wave function at the sphere
radius, or the E$_\nu $ of the wave function $\phi $.  For a given
potential, there is a unique correspondence between the logarithmic
derivative D$_\nu $ at the sphere radius and E$_\nu $, so in
principle, it is possible to specify either one.

There is a set of numbers called P (one for each $\ell $ channel) that
supplies the information about both the principal quantum number and the
logarithmic derivative, D. P is defined as:
P = .5 - atan(D$_\nu $)/$\pi $ + n where n (its integer part) is the
principal quantum number; its fractional part varies smoothly from 0
(for the bottom extreme of the band for that principal quantum number)
to 1 (the top extreme of the band), and can be thought of as a
`continuously variable' principal quantum number.  Here is a table of
P$_n$ as function of D$_\nu $: D$_\nu $ 10 5 1 0 -1 -2 -3 -4 -10 P$_n$
.03 .06 .25 .5 .75 .85 .90 .92 .97

P must be supplied to the atom program.  The fractional part of P is
automatically supplied by the output of a band calculation (provided
the token IDMOD in category CLASS is not 1),
but you must supply P (in addition to the moments Q) if
you choose to begin with moments.  P, together with the moments Q can be
input directly through the control file.  A word on choices for the
fractional part of P: .3 is quite free- electron-like and suitable for
free-electron-like $\ell $ channels such as Si d electrons, while .8 is
quite tight-binding-like and suitable for deep states like Cu d orbitals.
If there is no information from the very beginning, .5 is a suitable
choice.

In self-consistency cycles, you have a choice, through the parameter
IDMOD described below, regarding the related pair of parameters P and
E$_\nu $.  You may let P and E$_\nu $ float to the center of gravity
of the occupied part of the band (most accurate for self-consistent
calculations); this is the default.  You may also freeze alternatively P
or E$_\nu $ in the self-consistency cycle.

The problem of `ghost' bands can be overcome by using the downfolding
procedure.  Orbitals are separated in lower, intermediate and higher
sets. By switching on the automatic downfolding procedure,
this choice is done
automatically by the program.  Lower waves contribute to the dimension
of the Hamiltonian matrix, H, and the overlap matrix, O, and carry
charge.  Intermediate waves do not contribute to the dimension of H and
O, but carry charge.  Higher waves are thrown away altogether.  If it
is not known how to separate the orbitals the automatic downfolding can
be switched on before the self-consistency cycle.  After a few
iterations, the downfolding indices don't change and if it is desired
the cycle can be stopped and started again with the proper division into
lower, intermediate, and higher set.
Calculations using the automatic downfolding should be
checked carefully.  The token which makes the decomposition is
IDXDN.\\
IDXDN = 0 : automatic downfolding \\
IDXDN = 1 : low orbitals \\
IDXDN = 2 : intermediate orbitals \\
IDXDN = 3 : higher orbitals\\
More about this below under category CLASS and token IDXDN.

%\input{window}
\section{Accurate band structure away from the center of gravity}
\label{window}

Sometimes it is necessary to get accurate bands in a small energy range
e.g. around the Fermi energy, for an accurate description of the Fermi
Surface or in an energy range far from the center of gravity of the
partial waves.  In such cases the potential parameters have to be
recalculated with new E$_\nu $'s for the SCF potential.

After the self-consistent calculation which has generated atomic files
with potential parameters and potentials, one `iteration' is performed
with the following changes: In the CTRL file the E$_\nu $'s are
changed, and the tokens
NITATOM=1 and Q=BAND in category OPTIONS.  Furthermore, the tokens
BETA=0 BROY=F NIT=1 BEGMOM=T and CNTROL=F in category START.  With
these parameters {\em lm.run}
produces a BAND file with the new bands and eigenvectors, which may be
used for a DOS calculation.  It also produces atomic files with the new
potential parameters which may be used for calculation of energy bands
along symmetry lines.

Alternatively one can change the E$_\nu $'s in the CTRL file and run
{\em lmbnd.run}, since this will recalculate the potential parameters.
The Fermi energy is, however, not recalculated.

%\input{frozen}
\section{Frozen potential calculations}
\label{frozen}

In the ASA it is easy to calculate for instance structural energy
differences. This is done by the so-called frozen potential technique.
Assume that we want to calculate the structural energy difference
between fcc and bcc copper. First a self-consistent calculation for
fcc copper is performed in the usual way and the total energy and a
potential is obtained. The potential is inserted in the bcc structure
with the same atomic volume and the total energy is calculated without
iterating to self-consistency i.e. keeping the potential frozen. The
total energy difference in the two calculations is the structural
energy difference.

The second step is performed in the following way: The atomic file
containing the frozen potential and potential parameters is not changed,
but the CTRL file is.

In category START: ITER=1, so that only one iteration is performed
and BEGMOM=F, so that the program will start by making the hamiltonian
and overlap matrices. Eigenvalues and eigenvectors will be calculated
and from the latter the moments are obtained.

In category START: NMIX=0 so that linear mixing is effective and BETA=1
to ensure that the moments are not mixed with the previous ones.

In category CLASS: IDMOD=2 for all ${\ell }$ channels so that the
E$_{\nu }$'s are not shifted to the new center of gravity.

In category OPTIONS: NITATOM=0 so that only one incomplete loop in the
atomic program is performed. This produces the contribution to the
total energy from the sphere.


%\input{spin}
\section{Spin polarized calculations}
\label{spin}

A self-consistent spin-polarized calculation can be started just like a
non spin-polarized (hereafter called NM) calculation, except that
NSPIN=2 in Category OPTIONS
and the starting moments for up and down spins in Category START are
made different.

Often it is, however, faster first to perform a self-consistent NM
calculation and then using this to start the
spin-polarized calculation.

After the self-consistent NM calculation has been
performed the atomic files contain
the self-consistent moments, potential parameters and
potentials. These have to be doubled (one set for each spin)
and changed, before the spin-polarized iterations can begin.
This can be done by
changing NSPIN=1 to NSPIN=2 in Category OPTIONS and VERBOS=50
in Category IO and
running {\em lmctl.run}, which inserts the doubled moments (devided
by two) and potential parameters from the atomic files into the CTRL
file. The potential parameters or the moments then have to be changed
so that the two spin directions become different. Probably the most
predictable is to change the moments. Example: Assume paramagnetic Ni
has 9 $d$-electrons, then after {\em lmctl.run} the zeroth $d$ moments
for up and down spin are 4.5. We guess that the magnetic moment is
0.6 Bohr magnetons and the up-spin moment is changed to 4.8 and the
down-spin moment is changed to 4.2. The spin polarized iterations
are then started
using these values in the CTRL file by setting BEGMOM=T and CNTROL=T
in Category START.

The antiferromagnetic (AFM) calculation is different from the
ferromagnetic and antiferrimagnetic calculations because in the latter
two cases two band structure calculations have to be performed, one for
spin-up and one for spin-down. In the AFM case the two band structures
are identical, but the moments and potential parameters for one
sublattice and say spin-up are the same as those for the other
sublattice and spin-down. The program can take advantage of this by
performing only one band structure calculation and interchanging the
resulting moments. This is achieved by setting the token AFM=T in
category OPTIONS. It should be noticed that one is {\em not} free to
chose the order in which the atoms occur in the CTRL file. With the
atomic order in the first sublattice chosen, the order of the second
sublattice must be the same. Example: In NiO the spin-up sublattice
consists of Ni-spin-up and O-spin-up and the spin-down sublattice
consists of Ni-spin-down and O-spin-down. If Ni-spin-up is the first
and O-spin-up the second in the CTRL file, then Ni-spin-down must
be the third and O-spin-down must be the fourth atom. As described
above it may be advantageous to start the spin-polarized calculations
from a self-consistent NM calculation. This could be
done as indicated above, but since usually there are twice as many
atoms in the AFM case as in the NM case, it may be more convenient to
use the program {\em chspin.run} in the main directory to produce
`spin-polarized' atomic files from NM ones. To use
this, just type {\em chspin.run}. It will ask for the name of the
`non-spin-polarized' input file (let us call it NI) and the spin
direction in the output file (U, for up or D, for down). It will
produce a file with the name NIU or NID and with the same atom name in
the file (NIU or NID). In this file the moments, the potential
parameters, and the potentials are doubled (the moments divided by
two). The potential parameters or the moments then have to be changed
just as in the ferro-magnetic case described above. The extra atoms
are then inserted in category SITE and CLASS
BEGMOM=T and CNTROL=F in category START.



%\input{plot}

\section{Plot programs}
\label{plot}
All programs for simple two-dimensional plots are based on the public
domain program Gnuplot (Copyright (C) 1986 - 1993 Thomas Williams,
Colin Kelley), which is part of the LMTO distribution.

\subsection{Gnuplot}
\label{gnuplot}
Gnuplot is a command-line driven interactive plotting utility
for UNIX, MSDOS and VMS platforms.  The software is copyrighted but
freely distributed (i.e., you don't have to pay for it).  It was
originally intended for
visualization of mathematical functions and data.  Gnuplot
supports many different types of terminals, plottersx, and printers
(including many color devicesx, and pseudo-devices like LaTeX) and is
easily extended to new devices. The `GNU' in gnuplot is
NOT related to the Free Software Foundation, the naming is just a
coincidence (and a long story). Thus Gnuplot is not covered by the GNU
copyleft, but rather by its own copyright statement, included in all
source code.
The Gnuplot package comes with makefiles for the following UNIX platforms:
\begin{enumerate}
\item Apollo running SR10.3 with Apollo's X11
\item DEC3100/5000 running DEC OSF/1 v1.0
\item DEC3100/5000 running Ultrix 3.1d with MIT's X11
\item HP/9000 700 series running HP/UX 8.0 with MIT's X11R4
\item Sun sparcstation running SunOS 4.1 with suntools (with and without X11)
\item Silicon Graphics IRIS4D machines (with and without X11)
\item NeXT Cube and Slab running NeXTOS 2.0+ (with and without X11)
\item ATT 3b1 machines (with and without X11)
\item 386 machines running 386/ix or ISC 2.2 (with and without T.Roell X386)
\item IBM RS/6000 running AIX 3.2 with xlc 1.2 or xlc 1.1
\item Cray Y-MP or Cray-2 running Unicos 6.0 or 6.1 (with and without X11)
\item Sequent Dynix/PTX with MIT X11
\item Sequent Symmetry (DYNIX 3) with X11
\item Convex 9.0 with MIT X11
\item KSR1 running DEC OSF/1 v1.0 (use make -j 16)
\item LINUX with XFree86-1.2
\item VAX-VMS
\end{enumerate}
The following table  summarizes most of the terminals
(output formats) supported by Gnuplot Release~3.5.
\begin{tabular}{r l}
            table & Dump ASCII table of X Y [Z] values to \\
                  & output\\
            corel & EPS format for CorelDRAW\\
            eepic & EEPIC -- extended LaTeX picture \\
                  & environment\\
            emtex & LaTeX picture environment with \\
                  & emTeX specials\\
     epson-180dpi & Epson LQ-style 180-dot per inch \\
                  & (24 pin) printers\\
      epson-lx800 & Epson LX-800, Star NL-10, NX-1000, \\
                  & PROPRINTER ...\\
\end{tabular}
\begin{tabular}{r l}
          hp2623A & HP2623A and maybe others\\
           hp2648 & HP2648 and HP2647\\
          hp7580B & HP7580, and maybe other HPs (4 pens)\\
           hp500c & HP DeskJet 500c, [75 100 150 300] \\
                  & [rle tiff]\\
             hpgl & HP7475 and (hopefully) others (6 pens)\\
           hpljii & HP Laserjet series II, [75 100 150 300]\\
             hpdj & HP DeskJet 500, [75 100 150 300]\\
             hppj & HP PaintJet and HP3630 \\
                  & [FNT5X9 FNT9X17 FNT13X25]\\
\end{tabular}
\begin{tabular}{r l}
    kc/km-tek40xx & MS-DOS Kermit Tek4010 terminal \\
                  & emulator - color/monochrome\\
            latex & LaTeX picture environment\\
              mif & Frame maker MIF 3.00 format\\
          nec-cp6 & NEC printer CP6, Epson LQ-800 \\
                  & [monocrome color draft]\\
              pbm & Portable bitmap [small medium large] \\
                  & [monochrome gray color]\\
             pcl5 & HP LaserJet III [mode] [font] [point]\\
       postscript & PostScript graphics language \\
                  & [mode fontname font-size]\\
\end{tabular}
\begin{tabular}{r l}
          pslatex & LaTeX picture environment with \\
                  & PostScript specials\\
         pstricks & LaTeX picture environment with \\
                  & PSTricks macros\\
              qms & QMS/QUIC Laser printer \\
                  & (also Talaris 1200 and others)\\
            regis & REGIS graphics language\\
          tek410x & Tektronix 4106, 4107, 4109 and 420X \\
                  & terminals\\
          tek40xx & Tektronix 4010 and others; \\
                  & most TEK emulators\\
\end{tabular}
\begin{tabular}{r l}
          texdraw & LaTeX texdraw environment\\
             tpic & TPIC -- LaTeX picture environment \\
                  & with tpic specials\\
            vttek & VT-like tek40xx terminal emulator\\
          X11,x11 & X11 Window System\\
\end{tabular}

If there is access to one of the platforms listed above and one of the
supported terminals connected to it, all that needs to be done is
to compile
and install Gnuplot on the target machine (read the File 0INSTALL for
details).

\subsection {Common features of the plotting programs}

Five interface programs convert the output of the LMTO package to a format
suitable for Gnuplot and generate a command file where all options are set:\\[5mm]

\begin{tabular}{ l l }
{\bf gnucharge.run} & Charge density or ELF plot\\
{\bf gnupot.run}    & Hartree potential plot\\
{\bf gnubnd.run}    & Band structure plot\\
{\bf gnudos.run}    & Density of states plot\\
{\bf gnufs.run}     & Fermi surface plot\\[5mm]
\end{tabular}

These programs will be described in detail below. There are, however, some
common features which will be described first.
\par
Out of the variety of output devices and media, the most common ones are
supported in the programs. Starting one of the programs results in the
following prompt:
\begin{verbatim}
  Enter output device:
   1 = Postscript (default)
   2 = HP-GL pen plotter
   3 = HP Laserjet III (PCL5)
   4 = PC screen (vt220 emulation)
   5 = X-Windows
\end{verbatim}

Depending on the choice the following output is created:
\begin{enumerate}
\item A Postscript file named {\it charge.ps, pot.ps, bnds.ps, dos.ps}
 or {\it fs.ps}. This file can be send to any Postscript printer or
 be viewed
 with Display Postscript or Ghostview.
\item A file named {\it charge.hpgl, pot.hpgl, bnds.hpgl, dos.hpgl} or
 {\it fs.hpgl} suitable for all pen-plotters and laserprinters that understand
 HPGL (Hewlett-Packard Graphics Language).
\item A file named {\it charge.pcl, pot.pcl, bnds.pcl, dos.pcl} or
 {\it fs.pcl} suitable for all Hewlett-Packard laserprinters and compatibles.
\item Direct output to any vt220 compatible terminal including PCs
  emulating a vt220 (Tektronics 4014 and compatibles are also supported).
\item Direct output to a local or remote X-Window. If the
  program is used remotely, make sure that the remote host has
  permission to use
  the local display (xhost +localdisplayname) and set the environment
  variable DISPLAY to `localdisplayname:screennumber'.
\end{enumerate}

If the parameters have to be changed (e.g. Encapsulated Postscript instead
of Postscript) or another terminal supported by Gnuplot has to be
added, this can easily be done by
changing the source code where the line "set term ..." is written
to the Gnuplot command file. Refer to the Gnuplot manual for details.\\

All programs generate intermediate files containing the plot data and
a commandfile with all the commands and options necessary. The names of
these files are listed in the detailed description for each program below.
These files will remain on the disk if the executables (*.run) are used
directly, and that will allow some fine-tuning by editing the command
file (*.GNU).\\

For routine plots, however, it is better to run the shellscripts *.exec,
which call Gnuplot immediately after generating the plot data and delete
all intermediate files afterwards.

\subsection {Charge density and ELF plot}
\label{charplot}
The program {\it gnucharge.run} generates plots of the
calculated charge density or the Electron Localization Function (ELF).
The program reads the files RHO and ELF created by {\it lm.run} in charge mode
(CHARGE=T in category OPTIONS, ELF=T in category CHARGE) and creates the plot
data file CHARGE.DAT and the command
file CHARGE.GNU.
After the selection of the output device follows the question:
\begin{verbatim}
generate a charge density(t) or ELF plot(f)?
\end{verbatim}
Entering t (logical true) or / (End-of-record) selects a charge density plot,
whereas f (logical false) selects a plot of the ELF.
In the next step the plot data are read in and the minimum and maximum are
determined:
\begin{verbatim}
 Number of grid-points:  2821
 the charges range from .660E-03 to .581E+01
\end{verbatim}
By default the plot range extends from the minimum to the maximum of the data.
However, you can modify the range by entering two scalars bounding the
new interval:
\begin{verbatim}
  enter new min, max or / for default
\end{verbatim}
The following question concerns the number of isocontour lines to be drawn
\begin{verbatim}
 how many contours?(default is 11)
\end{verbatim}
You may enter any number between 2 and 98 or / to accept the default.
The iso contours are now equally spaced from minimum to maximum including
the upper and lower bound. Assuming 11 lines for a minimum of 0.0 and a maximum
of 1.0 results in contours representing the values 0, 0.1, 0.2 ... , 0.9, 1.0.
The next question defines the nature of the plot:
\begin{verbatim}
 draw a 3d-surface(t) or not(f)?(default: f)
\end{verbatim}
Entering t (logical true) places a 3d surface with hidden-line-removal on top
of the contour plot, whereas f (logical false) or / lead to a simple
contour plot.
Now the plotfile or the plot window should be generated.\\

{\it Important notice: Gnuplot does not retain proper x-y-scaling for most
     output formats unless your plot is quadratic. You generally have
     to scale the plot manually by editing the directive ''set size ...''
     in CHARGE.GNU.}

\subsection {Hartree potential plot}
\label{hartplot}

The program {\it gnupot.run} generates plots of the superimposed atomic
Hartree potentials of the free atoms. This plot can be useful to check
the estimate of the sphere radii performed by {\it lmhart.run}.
The program reads the file POT created by {\it lmhart.run} for the plane
specified in the PLOT category in the CTRL file and creates the plot data
file GPOT.DAT and the command file POT.GNU.
After the selection of the output device the plot data are read in and the
minimum and maximum are determined:
\begin{verbatim}
lines:  2700
time for reading pot:  1.85
the potential range is from -.1E+7 to -.829E-1
\end{verbatim}
By default the plot range extends from the minimum to the maximum of the data.
Since these potentials usually cover several orders of magnitude, it is
advisable to modify the range by entering two scalars bounding the
new interval (e.g. -5.0 0.0) after this prompt:
\begin{verbatim}
  enter new min, max or "/" for default
\end{verbatim}
The following question concerns the number of isocontour lines to be drawn
\begin{verbatim}
 how many contours? (default is 11)
\end{verbatim}
You may enter any number between 2 and 98 or / to accept the default.
The iso contours are now equally spaced from minimum to maximum including
the upper and lower bound as explained in the {\it gnucharge} section.
The next question defines the nature of the plot:
\begin{verbatim}
 draw a 3d-surface(t) or not(f)?(default: f)
\end{verbatim}
Entering t (logical true) places a 3d surface with hidden-line-removal on top
of the contour plot, whereas f (logical false) or / lead to a simple
contour line plot.
Finally you are prompted for a 40-character title string for the plot:
\begin{verbatim}
 enter plot title :
\end{verbatim}
Now the plotfile or the plot window should be generated.\\
As stressed in the {\it gnucharge} section, Gnuplot may change the x-y-scaling.

\subsection {Band structure plot}
\label{bandplot}

The program {\it gnubnd.run} generates plots of the band structure along the
symmetry lines defined in the SYML category of the CTRL file.
The program reads the CTRL file, the BNDS file and, if FATBAND=T in the
category options in the CTRL file,
the file EIGN. The latter two are created by
{\it lmbnd.run}. It creates the plot data files BNDS.DAT, FERMI.DAT and the
command file BNDS.GNU. For an orbital projected bands plot BNDS2--4.DAT
are created in addition.

After the selection of the output device one is prompted for a
40-character title string for the plot:
\begin{verbatim}
 enter title:
\end{verbatim}
followed by the definition of the energy unit:
\begin{verbatim}
 energies in Ryd.(f) or eV(t)? (default f)
\end{verbatim}
Entering t (logical true) selects electron volts, whereas f (logical false) or
/ selects Rydberg. The next prompt permits you to shift the origin of
the energy scale to the Fermi energy:
\begin{verbatim}
  energies relative to EF (t)? (default f)
\end{verbatim}
Depending on the choice of symmetry lines and energy range it may be
advantageous to plot the band structure in portrait mode instead of the
default landscape mode:
\begin{verbatim}
  landscape plot (t) ? (default t)
\end{verbatim}
To achieve this enter f (logical false) after this prompt. If
subsequent points of the same band should not be connected by a line,
enter f (logical false) at the next prompt:
\begin{verbatim}
  energies connected by lines (t)? (default t)
\end{verbatim}
The decision whether orbital projected bands are to be plotted or not is
the next issue:
\begin{verbatim}
  plot orbital character(t)? (default f)
\end{verbatim}
Enter t (logical true) to plot orbital projected bands, whereas f
(logical false) or / result in an ordinary band plot. If the fat bands
option is not used you may skip to the modification of the energy range
below. If the fatband plot option is chosen, the prompted is:
\begin{verbatim}
Change coordinate system? Enter Euler angles:
alpha, beta, gamma(units of Pi).
If nochange: enter "/"
\end{verbatim}
If the internal coordinate system should be changed, enter the Euler
angles$^{11}$.
Enter / to retain the current orientation. Now the atoms
whose orbitals should be displayed fat should be given:
\begin{verbatim}
no coordinate transformation!
Enter orbital character plotted as "fatbands"
First, enter name(s) of atom(s) format(20a4)
followed by <enter> and "/" on next line
\end{verbatim}
Enter the names of the atoms as specified in the CLASS category of the CTRL
file separated by blanks. Enter `/' on the next line. If there are several
atoms of this class, can be selected:
\begin{verbatim}
RB1
Following atom(s) selected RB1
There are  6  atoms of type RB1
Specify which one(s) to select, e.g. 1 3 7 for
the first, third, and seventh, followed by "/"
\end{verbatim}
Enter a list of integers followed by `/'. Next the orbitals
to be included in the fatband plot should be given:
\begin{verbatim}
For each atom specify orbital number from list:
NB! Orbitals are in the new coordinate system!
s  y  z  x  xy  yz  3z^2-1  xz  x^2-y^2
1  2  3  4   5   6     7     8     9
y(3x^2-y^2) xyz y(5z^2-1) z(5z^2-3)
     10      11   12        13
 x(5z^2-1) z(x^2-y^2) x(3y^2-x^2)
     14         15          16
RB1 number 1 enter number(s) followed by "/":
\end{verbatim}
Again, enter a list of integers followed by `/'. This step is repeated for
all selected atoms of all selected classes.
Now all fat band specific options have been set. The input files are read in
and some information about them is displayed:
\begin{verbatim}
 Bands= 44 Fermi Energy= -.1426
 Lattice const.=29.589 Spins=1
 nq1=   35
 nq1=   40
 nq1=   20
 nq1=   30
 nq1=   25
 nq1=   45
 nq1=    0
 ebot=-.3864 etop=.3131 eferm=-.14258
   nq=  195   nline=    5
 default emin and emax =  -1.0 1.0
\end{verbatim}
If band structure plot should be limited to a fraction of the
calculated energy range the new boundaries may be entered after this
prompt:
\begin{verbatim}
 enter emin, emax
\end{verbatim}
Entering / instead of two scalars selects the whole range shown above.
Now the plotfile or the plot window should be generated.\\

{\it Important notice: Gnuplot is currently not capable of handling more than
one font per plot. Therefore it is impossible to mix greek and roman
letters and e.g. the $\Gamma$ point is represented by $G$.
If the output is postscript, however, it is possible to edit the file
bnds.ps and replace the sequence\\ {\tt ( G ) Cshow}\\ by \\
{\tt /Symbol findfont 180 scalefont setfont\\ ( G ) Cshow\\ /Times-Roman
findfont 180 scalefont setfont}.

One can  move the label of the
y-axis closer to the plot, simply by changing the line\\
{\tt currentpoint gsave translate 90 rotate 0 0 moveto ( Energy \(eV\)) Cshow}\\
to\\
{\tt currentpoint gsave translate 90 rotate 0 -700 moveto ( Energy \(eV\)) Cshow}.\\
}
\subsection {Density of states plot}
\label{dosplot}
The program {\it gnudos.run} generates total or partial density of states plots.
The program reads the file DOS created by {\it lmdos.run} and creates the plot
data files DATA.DOS, FERMI.DOS and DOS.GNU.
After the selection of the output device one is prompted for the energy
units to be used:
\begin{verbatim}
 energies in Ryd.(f) or eV(t)? (default f)
\end{verbatim}
Entering t (logical true) selects electron volts, whereas f (logical false) or
/ selects Rydberg. The next prompt allows a shift of the energy
origin to the Fermi energy:
\begin{verbatim}
  energies relative to EF (t)? (default is f)
\end{verbatim}
Now the DOS file is read and information about it's content is displayed:
\begin{verbatim}
 emin,emax= -1.0 .0, nopts=1001,
 nd=24 efermi=-.1426

 classes are: NA1  NA2  RB1  RB2  SB1  SB2
 l's     are: s  p  d  f
\end{verbatim}
Next the projected DOS which should be added and plotted are selected:
\begin{verbatim}
 Enter class1-l1 class2-l2 .. to be added to DOS
 Examples:   all   s   p   NA1-f  SB2-s
\end{verbatim}
If `all' is entered, then the total density of states is plotted. Entering
the name of one or several atoms yields a plot of the partial density
of states associated with these atoms. Entering the orbital character
(s,p,d,f) makes a $\ell $-projected DOS which is summed over all atoms.
If the name of a class of atoms augmented by  `-' and an orbital character
symbol is entered, only the partial density of states associated with these
atoms and this specific $\ell $-value is plotted. Additive combinations
are also possible ( e.g. RB1 + s generates a plot of all s-orbitals
present and all additional orbitals on the RB1 atoms).\\
Following a table of all selected orbitals, a choice of the energy
range is prompted:
\begin{verbatim}
Take: NA1-s NA1-p NA1-d NA2-s NA2-p NA2-df

\end{verbatim}
Following the table of all selected orbitals you are prompted for the
relative weight of each partial DOS:
\begin{verbatim}
 Now enter weights for partial DOS
 (default=1,1,..)
 A weight of 1.0 for each partial DOS gives
 the correct total DOS.
\end{verbatim}
Entering / for the default yields the correct total DOS, but you may enter a
number for each partial DOS.
Next you are prompted for the energy range:
\begin{verbatim}
 emin  , emax   =  -1.000    .000
 if desired, enter new emin emax, / for default
\end{verbatim}
Entering / instead of two numbers selects the whole range shown above.
The energy units are the ones selected in the first step. The default
range is defined in the DOS category in the CTRL file.
The next prompt permits changing the range of the density of states
in exactly the same way:
\begin{verbatim}
 dosmin, dosmax =.0 1070.419
 if desired, enter new min max,/ for default
/
  new values for emin, emax, dmin, dmax are:
    -1.000       .000      .000     1080.0

 title = all
 if desired, enter new title / for default
\end{verbatim}
Finally a title for the plot can be given:
Now the plotfile or the plot window should be generated.
\subsection {Fermi Surface plot}
\label{fsplot}
{\it Important notice: This program is not yet fully tested for all lattices!
     Please cross-check the results carefully!}\\

The program {\it gnufs.run} generates a plot of Fermi surface
contours in the unfolded
irreducible part of the Brillouin zone.
The program reads the file LMFS created by {\it lm.run} if FS=T in category
OPTIONS and creates the plot data files Y1.DAT, Y2.DAT and the command
file Y.GNU.
After the selection of the output device one is prompted for a
40-character title string for the plot:
\begin{verbatim}
 enter title:
\end{verbatim}
followed by a list of lattices and the prompt for a general
scaling factor and the selection of the lattice:
\begin{verbatim}
Number  Lattice type
1       sc
2       bcc
3       fcc
4       orthor primitive
5       orthor based centered
        different setting for 1248
6       orthor primitive,diff. sett.
7       orthor based centered-real
8       tetrag primitive kz=0 plane
x9      tetrag primitive more planes
        x=1->4 for different planes
enter: scale and lattice
      e.g.: 8.  4 (orthorhombic)
      or    3.  8 (tetrag)
      or    4. x9 (tetrag for several planes,
               x selects which one)
\end{verbatim}
After the scale and the lattice is entered, a short summary of the
input is printed and the plot is generated.

%\input{copyri}
\section{Copyright}
\label{copyri}
\noindent
 The Stuttgart Tight-Binding LMTO-ASA program Version 4.7    \\
    Copyright (C) 1998              \\
\indent
      Max-Planck-Institut f\"ur Festk\"orperforschung        \\
\indent
      O. K. Andersen                                         \\
\indent
      Heisenbergstrasse 1                                    \\
\indent
      D-70569 Stuttgart                                      \\
\indent
      Germany                                                \\

\section{License Agreement}
\label{license}
\noindent
All programs are subject to a license agreement. Be sure to read and 
agree to the license BEFORE you use the programs. 
\newline
\newline
\noindent
INSTALLING THE PROGRAMS, YOU ARE CONSENTING TO BE BOUND BY AND ARE 
BECOMING A PARTY TO THIS AGREEMENT. IF YOU DO NOT AGREE
TO ALL OF THE TERMS OF THIS AGREEMENT, DO NOT INSTALL THE PROGRAMS. 
\newline
\newline
\noindent
LIMITATIONS ON USE 
\newline
\newline
\noindent
All programs are a free software for scientific and/or educational
purposes. You are not allowed to redistribute it without prior written
consent of the Copyright owner. It is
illegal to commercially distribute these programs as a whole or
incorporate any part of them into a commercial product. 
\newline
\newline
\noindent
DISCLAIMER OF WARRANTY 
\newline
\newline
\noindent
The programs are provided on an "AS IS" basis, without warranty of any
kind, including without limitation the warranties that they are free of
defects and fit for a particular
purpose. The entire risk as to the quality and performance of the
programs is borne by you. Should the programs prove defective in any
respect, you and not Licensor assume
the entire cost of any service and repair. 

\appendix
%\input{prepro}
\section{The preprocessor}
\label{prepro}

A preprocessor {\em ccomp}
provides a simplified, FORTRAN compatible version
of C conditional compilation.  FORTRAN statements beginning with C\# are
preprocessor directives; the ones implemented now are C\#ifdef,
C\#ifndef, C\#else, C\#elseif, C\#endif (also C\#define defines a name).
Directives C\#ifdef, C\#ifndef, C\#elseif, and C\#endif are followed by
a name, e.g.  C\#ifdef CRAY. when C\#ifdef is false (either name is not
defined or it lies within an \#if/\#endif block that is false),
{\em ccomp}
comments out until a change of state (new C\#ifdef, C\#ifndef, C\#else,
C\#elseif, C\#endif encountered); C\#ifdef is true, {\em ccomp}
uncomments
lines following until another conditional compilation directive is
encountered.

Conditional compilation blocks may be nested.  As with C, {\em ccomp}
distinguishes case.  Output is to standard out.

There is a primitive facility to make logical expressions using the AND
(\&) and OR ($|$) operators, such C\#ifdef john \& bill, or C\#ifdef
john $|$ bill, is provided.  Precedence of operators is strictly left to
right, so that john $|$ bill \& mike is equivalent to (john $|$ bill) \&
mike, whereas john \& bill $|$ mike is equivalent to (john \& bill) $|$
mike

How {\em ccomp} determines whether to modify code:

Whether the lines following a C\#ifdef, C\#ifndef, C\#else, C\#elseif,
C\#endif need to be commented out or uncommented depends on whether they
have been commented out in a previous pass.  This information is
contained in a `C' following the directive, e.g.  C\#ifdefC, C\#ifndefC,
C\#elseC, C\#elseifC, C\#endifC.  The preprocessor will set this, it is
best advised to create any new blocks uncommented and let the
preprocessor do the commenting.

The main programs in directory MAIN: {\em lm.f, lmstr.f, lmbnd.f,
lmdos.f, lmovl.f}, and {\em lmhart.f} are all obtained from
and {\em lmall.f} in the main directory, by running the latter
{\em ccomp} with different
keywords defined.  To modify a main program change {\em lmall.f} and
execute {\em make all}.  This will create all the main programs in MAIN
and compile and link them.  Similarly for the atomic program a
non-relativistic version may be generated using the keyword NONREL.

The following line illustrates the use of the preprocessor (after it has
been compiled by the C-compiler) IBM (CMS) machine: ccomp -uLM -dLMSTR
LM LMSTR. `-uLM' means undefine LM i.e. remove LM specific lines.
`-dLMSTR' means define LMSTR i.e. uncommend LMSTR specific lines. `LM'
is the filename of the input file with file type FOR and file mode A.
`LMSTR' is the output file.  Multiple defines and undefines are
possible.
%\input{machines}
\section{Machine Dependencies}
\label{machines}

The LMTO programs have been written with portability in mind, and some
effort was made to keep within ANSI-77 standards.  If you do find any
compilation errors in the code as it stands, please report them, so that
the code can be made as portable as possible.  There is some system
dependence lying outside the purview of ANSI.  File handling is very
operating-system dependent, and LMTO files have been kept as simple as
possible to make the code as portable as possible.  All files are opened
in function {\em fopna} in directory IOLIB,
so whatever machine dependence there is
should be confined to that subroutine.  Also, different machines have
differing internal representations of their numbers.  In some places
this is important, such as the function {\em derfc}.
There is a collection of
three functions in the math library, {\rm r1mach, d1mach) and
{\em i1mach} that hold
machine-specific information.  These must be set to the appropriate
machine before compilation.  Besides the four routines,
{\em fopna, i1mach, r1mach} and
{\em d1mach}, a few occasional machine-dependencies are found.
Perhaps the most important exception occurs in routine {\em cnvt} in
the IOLIB directory.
There data read from the control file is converted from ASCII
representation to the binary representation, using the FORTRAN internal
read facility.  This read differs on different machines.  The
unformatted read should be used if possible; if not an alternate
formatted read is supplied.  This should be checked when installing this
routine on a new compiler.  Routine {\em finits} (also in IOLIB) does
some machine-dependent initialization.

\subsection{Machine constants}

The following machine dependent constants are used in the TB-LMTO prog.
They are supplied in functions in {\em extens.f}:

dmach(1)=1.d-99 [ smallest double precision number]

dmach(2)=1.d+99 [ largest double precision number]

dmach(3)=1.d-15 [ smallest number which added to 1.0 gives something
different from 1.0 ]

imach(2)=6 [ standard output channel ]

imach(4)=6 [ standard error channel ]

imach(6)=4 [ no. characters per integer ]

imach(17)=1 [ no. integer words per real word ]

imach(18)=2 [ no. integer words per double precision word ]

%\input{ref}
\begin{references}

\bibitem{LMTO} O.K. Andersen, Phys. Rev. B {\bf 12}, 3060 (1975);
O.K. Andersen, Europhysics News {\bf 12}, 4 (1981);
O.K. Andersen, in {\it The Electronic Structure of Complex Systems},
edited by P. Phariseau and W.M. Temmerman (Plenum Publishing
Corporation, 1984); O.K. Andersen, O. Jepsen, and M. Sob, in
{\it Electronic Band Structure and Its Applications}, edited by
M. Yussouff (Springer-Verlag, Berlin, 1986); H.L. Skriver, {\it
The LMTO Method}, (Springer-Verlag, Berlin, 1984).

\bibitem{TBLMTO}  O.K. Andersen and O. Jepsen, Phys. Rev. Lett.
{\bf 53}, 2571 (1984);
O.K. Andersen, O. Jepsen, and D. Gl\"otzel, in
{\it Highlights of Condensed-Matter Theory}, edited by F. Bassani,
F. Fumi, and M.P. Tosi (North-Holland, New York, 1985);
O.K. Andersen, Z. Pawlowska, and O. Jepsen, Phys. Rev. B {\bf 34%
}, 5253 (1986);
H.J. Nowak, O.K. Andersen, T. Fujiwara, O. Jepsen, and P. Vargas, Phys
Rev. B {\bf 44}, 3577 (1991).

\bibitem{DOWNF} W.R.L. Lambrecht and O.K. Andersen, Phys. Rev.
B {\bf 34}, 2439 (1986); O.K. Andersen, T. Paxton, O. Jepsen, and M. van
Schilfgaarde (to be published).

\bibitem{tetra} O. Jepsen and O.K. Andersen, Solid State Commun.
{\bf 9}, 1763 (1971) and Phys. Rev. B {\bf 29}, 5965 (1984);
P. Bl\"ochl, O. Jepsen, and O.K. Andersen, Phys. Rev. B {\bf 49},
16223 (1994).

\bibitem{Boston} O.K. Andersen, A.V. Postnikov, and S.Yu. Savrasov,
Mat. Res. Symp. Proc. {\bf 253}, 37 (Materials Research Society 1992)
Chapter V.

\bibitem{LnI}  O. Jepsen and O.K. Andersen, Z. Phys. {\bf B 97},
35 (1995).

\bibitem{Symmetry}  C.J. Bradley and A.P. Cracknell, {\it The
Mathematical Theory of Symmetry in Solids} (Clarendon, Oxford, 1972).

\bibitem{Spacegr}  {\it International Tables for Crystallography},
edited by Theo Hahn (Riedel, Boston, 1987) Vol. A

\bibitem{Pearson}  {\it Pearson's Handbook of Crystallographic
Data for Intermetallic phases}, by P. Villars and L.D. Calvert
(American Society for Metals, Metal Park, Oh44073)

\bibitem{AJK}   O.K. Andersen, O. Jepsen, and G. Krier;
Exact Muffin-Tin Orbital Theory. In METHODS OF ELECTRONIC STRUCTURE
CALCULATIONS edited by V. Kumar, O.K. Andersen, and A. Mookerjee
(World Scientific, Singapore, 1994) p. 63

\bibitem{Tinkham}  M. Tinkham, {\it Group Theory and Quantum
Mechanics} (McGraw-Hill Book Company, New York, 1964) p. 102

\end{references}

\end{document}
